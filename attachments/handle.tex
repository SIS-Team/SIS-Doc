\chapter[Handle]{Vertiefende Aufgabenstellung: \\Handle Marco}

\section{Zusammenfassung}
Meine Aufgabe in diesem Projekt ist zum größten Teil darin, die Eingabemasken für sämtliche Eingaben zu generieren. Weiteres bestand meine Aufgabe darin, die Datenbank anzulegen, alle Tabellen und Spalten zu erstellen und diese, falls notwendig, zu verändern. Die Datenbanken mit Daten zu befüllen und diese Daten zu verwalten wurde ebenfalls von mir vorgenommen. Weiteres half ich mit, die App Windows Phone tauglich zu machen.\\
Beim zusätzlichen FTKL-Projekt war Matthias Klotz und ich für den größten Teil der Hardware-Arbeiten zuständig. Wir nahmen auch die Installationen der Hardware an den Monitoren vor.\\
Beim Punkt Logging übernahm ich ebenfalls das Erstellen der Tabellen und anschließend das Generieren der Statistiken.\\
\\
Die größten Schwierigkeiten, die sich in meinem Teil zeigten, waren die Gebiete, die mit dem Thema Supplierungen zu tun hatten. Dabei stellten sich im Laufe des Projektes soviel Sachen heraus, auf die Rücksicht genommen werden musste.\\
Ein weiteres Problem stellte dar, die optimale Eingabemaske für die Eingabe der Stundenpläne zu finden. Dies konnte jedoch in Zusammenarbeit mit Herrn Stecher gelöst und ein optimaler Weg gefunden werden. 
\section{Projekterfahrung}
Das Projekt machte mich um viele Erfahrungen reicher. Vor allem, was das Thema Sicherheit auf Webseiten anbelangt, konnte ich einiges lernen. Ein ebenfalls wichtiger Punkt ist die Erfahrung, das erste Mal mit 3 weiteren Personen ein so umfangreiches Projekt in Angriff zu nehmen. Das Projektmanagement mit dem Tool Github kannte ich zuvor auch nicht. Ich würde sagen, dieses Tool hat die Zusammenarbeit um ein Vielfaches vereinfacht.\\
Auch die enge Zusammenarbeit mit der Schule war Neuland für mich. Denn ich würde sagen, unsere Diplomarbeit ist direkt verbunden mit der Schule, weshalb ein so direktes Zusammenarbeiten unausweichlich ist. Ich muss sagen, ich konnte keine negativen Erfahrungen machen.\\
Eine Erkenntnis reicher bin ich auch, wenn es um das Thema App Stores geht. Vor allem der Apple Store. Wie lange so etwas dauern kann und wie umfangreich so eine Abgabe sein muss, damit die App in die Stores aufgenommen wird, war uns allen nicht bekannt.\\
Durch die teilweise Grübeleien mit Matthias Klotz wegen der Windows Phone App habe ich auch erfahren, wie schwierig es sein kann, eine App für mehrere Plattformen zum Laufen zu bringen. \\


\chapter[Klotz]{Vertiefende Aufgabenstellung: \\Klotz Matthias}

\section{Zusammenfassung}
Meine Aufgabe bei diesem Projekt bestand darin, eine Smartphone-App für SIS zu entwickeln. Dabei müssen wie bei der Webseite auch, der Stundenplan, der Supplierplan, ein angepasster Stundenplan und die News angezeigt werden. Die App soll für Android iOS und WindowsPhone erstellt werden. Deshalb wird die App zuerst mit Webtechnologien erstellt und dann mit einem Framework für alle drei Systeme kompiliert.\\
Des Weiteren musste ich alle drei Apps in die jeweiligen App Stores laden. Dabei musste ich mit einigen Lehrern zusammenarbeiten, damit diese die App auch in den nächsten Jahren noch updaten können und damit die Bezahlung der Developer-Accounts von der Schule übernommen wird.\\
Bei dem zugehörigen FTKL-Projekt, machten Marco Handle und ich die Arbeiten an der Hardware und installierten diese auch bei den Monitoren im Schulgebäude.\\
Als große Schwierigkeit stellte sich das Veröffentlichen der App in den App-Stores heraus. Bei den ersten Versuchen wurden oft Fehlermeldungen zurückgesendet, welche nach mehreren Versuchen behoben werden konnten. \\

\section{Projekterfahrung}
Bei diesem Projekt arbeitete ich das erste Mal mit drei anderen Schülern an einem Projekt dieser Größe. Dabei erlernte ich den Umgang mit der Versionsverwaltungssoftware GitHub, welche den Austausch des Programmcodes enorm erleichterte.\\
Ich erkannte während des Projektes auch, dass man oft länger an einem Problem hängen bleibt und teilweise sehr kreativ werden muss, um gewisse Probleme zu lösen, da man die Lösung nicht einfach in einem Buch oder im Internet nachschlagen kann.\\
Eine andere Erfahrung bei diesem Projekt, war das Veröffentlichen der App in den verschiedenen App-Stores. Es war erstaunlich wie viele Angaben man zu der App machen muss, um die Applikation veröffentlichen zu dürfen.\\
Dabei waren vor allem die Unterschiede bei den verschiedenen Stores erstaunlich, da bei manchen Stores sehr viele Angaben zur App gemacht werden mussten und die Veröffentlichung dann auch sehr lange dauerte, wohingegen bei einem anderen Store die Veröffentlichung der App innerhalb weniger Stunden abgehandelt war.\\

\section{Entwicklungswerkzeuge}
Für die Entwicklung wurde der Editor Sbulime Text 2 (Version 2.0.2) verwendet. Um die entwickelten Seiten zu testen wurde ein Webbrowser verwendet (hauptsächlich Google Chrome Version 30.0.1599.69). Bei dem Betriebssystem, auf welchem gearbeitet wurde, handelt es sich um Windows 7 Service Pack 1.
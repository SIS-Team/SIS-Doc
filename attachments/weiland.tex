\chapter[Weiland]{Vertiefende Aufgabenstellung: Weiland Mathias}

\section{Zusammenfassung}
Ich war bei diesem Projekt hauptsächlich für die Ausgaben auf der Website und den Stundenplan, die News und den Supplierplan auf den Monitoren zuständig, aber auch für die Eingabe der News. \\\\
Beim FTKL-Projekt war ich hauptsächlich an der Leiterplattenfertigung beteiligt, die jedoch aufgrund mehrerer Design- und Produktionsfehler mehrmals durchgeführt wurde.\\\\
Die größten Schwierigkeiten, bereiteten der modifizierte Stundenplan und die PDF-Ausgabe. Beim modifizierten Stundenplan gab es so viele verschiedene Fälle, dass ich selbst teilweise die Übersicht verlor und die PDF-Ausgaben so hinzubiegen, das sie nicht zu schlimm aussehen und vor allem alles anzeigen, war fast genauso schwer.
\section{Projekterfahrung}
Gerade zu Beginn war das Projekt eine große Herausforderung, da ich schon seit Jahren nichts mehr mit PHP gemacht habe und die meisten anderen verwendeten Sprachen anfangs gar nicht beherrschte. Aber dank der Hilfe meiner Mitarbeiter war das kein Problem. Auch hab ich vorher noch nie an einem Projekt mit diesem Ausmaß teilgenommen.\\\\
Am meisten hab ich bei der Dokumentation der möglichen Sicherheitsprobleme gelernt. Ich wusste nicht, dass es für Hacker so leicht ist, an sensible Daten zu gelangen, wenn der Besitzer der Website auch nur eine Kleinigkeit übersieht.\\\\
Das Projekt war zwar teilweise stressig (besonders in der letzten Woche), aber abschließend kann ich nur sagen, dass das gesamte Projekt eine großartige Erfahrung war und ich froh bin, bei etwas mitgearbeitet zu haben, was hoffentlich noch von vielen Schülergenerationen und Lehrern verwendet wird.
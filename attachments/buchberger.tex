\chapter[Buchberger]{Vertiefende Aufgabenstellung:\\Buchberger Florian}

\section{Zusammenfassung}
Zu meinen Aufgaben gehörte das Design der Datenbank, der Entwurf und die Implementierung des Basissystems, die Anbindung des Systems an das LDAP-Cluster der Schule, der Entwurf des Monitor-Systems und dessen teilweise Programmierung. Auch noch andere Dinge, wie etwa der Schutz gegen XSRF, oder das Session-Management waren Teil meines Aufgabengebietes. Die Routinen zum Mitschreiben der Seitenaufrufe sind ebenfalls von mir entworfen worden. Das von Philipp Machac entworfene Design wurde von mir implementiert.\\
Außerdem übernahm ich die Basiskonfiguration der Raspberry Pis, sowie die Implementierung der jeweils notwendigen Software sowohl für SIS, als auch für das dazugehörige FTKL-Projekt.\\
Im Zuge der Projektplanung und Verwaltung wurde von mir Github als Versionsverwaltung und \LaTeX  als Dokumentations-Sprache vorgeschlagen.\\
\\
Die einzigen Probleme traten beim Design auf, da dieses unmittelbar mit dem Benutzer in Kontakt steht.
\section{Projekterfahrung}
Die wohl wichtigste Erfahrung dieses Projekts war die Zusammenarbeit mit weiteren Mitarbeitern. Auch konnte ich Wichtigkeit von Projektplanung und -management fühlen und werde wohl in Zukunft mehr wert darauf legen.\\
Teilweise war es auch anstrengend. Auch wenn ich nicht direkt daran beteiligt war, ist wohl das beste Beispiel das Hinzufügen der App zu den App-Stores.

\section{Entwicklungswerkzeuge}
Für die Code-Entwicklung wurde von mir hauptsächlich der Editor kate aus dem KDE-Packet in der Version 3.11.5 verwendet. Für die Vorschau des Codes wurde direkt der Webbrowser verwendet (Chromium in der Version 34.0.1847.116).

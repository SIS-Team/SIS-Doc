Für die Umsetzung wurden handelsübliche Funksteckdosen verwendet, da diese relativ günstig in den meisten Geschäften zu haben sind. Diese erfüllen jedoch Standardmäßig einen anderen Zweck, diese werden über eine Fernbedienung ein bzw. aus geschaltet. Da es uns nicht möglich ist, Steckdosen mit dem Raspberry über Funk zu steuern, musste eine andere Möglichkeit her.\\
Uns kam die Idee, dass in diesen Steckdosen eigentlich nur ein Relais vorhanden sein muss, welches die 230V schaltet. Diese Tatsache machten wir uns zu nutze und suchten auf der Platine nach diesem Schaltpunkt und entwickelten eine passende Schaltung, welche diese Relais schaltet.\\
Um widerrechtliches steuern der Steckdosen durch gefinkelte Schüler zu verhindern und da das Funkmodul nur auf die Platine aufgelötet war, bauten wir diese Platine aus. Somit ist es nicht mehr möglich die Steckdosen mit der passenden Fernbedienung zu steuern.\\\\
Die Schaltung der Steckdose ist in keinster Weise dokumentiert, auch die auf der Schaltung vorhandenen IC's sind kaum bis gar nicht dokumentiert, weshalb es uns nicht möglich war die Funktionsweise der einzelnen Teile fest zu stellen. Durch Analysen der Leiterbahnen konnten wir uns eine grobe Übersicht über die Schaltung verschaffen. Dieser Überblick reichte jedoch nicht aus, um den Schaltpunkt des Relais zu finden. Der eigentliche Schaltpin am Relais konnte identifiziert werden, jedoch schaltet dieses Relais erst bei 48V, welche nicht angelegt werden konnte, deshalb musste eine andere Möglichkeit gefunden werden.\\ 
Die Art des Suchens war nicht sehr professionell, da wir versuchten, durch Anlegen einer Steuerspannung an bestimmte Pins der ICs, den Pin zu finden, durch welchen das Relais geschaltet wird. Nach einigen Versuchen konnte der Punkt gefunden werden, dieser war doch nicht ideal, da dessen Innenwiderstand sehr klein war, was sich dahingehend zeigte, dass der Strom, der benötigt wurde, um die Steuerspannung zu halten, sehr hoch war. Wir fanden allerdings einen besseren Punkt, welcher wesentlich weniger Strom zog. Durch das Anlegen von 5V schaltete das Relais, dies entspricht der Versorgungsspannung des Funkmoduls. Es liegt nahe, diese zu verwenden.\\
Für eine leichtere Montage wurden die Vorhandenen Kontakte des Funkmoduls verwendet. Über die Kontakte wurde die Platine versorgt, diese Funktionsweise verwendeten auch wir, weshalb wir von dem gefundenen Punkt eine Drahtbrücke zum Kontakt machten. Die 5V waren sowieso schon auf einem der Kontakte vorhanden. Um die ursprüngliche Funktion des Kontaktes ab zu schalten trennten wir durch Entfernung eines Widerstandes die Leiterbahn auf. Nun musste nur mehr die Platine in die Kontaktlöcher gesteckt werden. Die nötigen Signale sind bei den Kontakten angelegt und werden über die Verbindung auf der Platine bereitgestellt.\\
Nun musste noch eine Lösung gefunden werden, wie mit dem Raspberry diese 2 Punkte verbunden werden konnten. Dies wurde mit einer Transistorschaltung realisiert. Wenn der Raspberry an einem der GPIO-Pins digital 1 (analog 3,3V DC) anlegt, so schaltet der Transistor und verbindet somit die zwei Punkte. \\
Dies war jedoch nicht genug, da noch eine Potentialtrennung zwischen Raspberry und Steckdose  notwendig war, wie ein Test ohne Potentialtrennung zeigt. (Siehe Punkt???). Diese lösten wir mit einem Optokoppler, welcher den Raspberry von der Steckdose trennen soll. Dies erfolgte mit der Schaltung ordnungsgemäß. Jedoch packten wir die Potentialtrennung auf den Header, welcher auf den Raspberry aufgesteckt wird, dies stellte sich als fataler Fehler heraus. (siehe Punkt ????)\\
Der finale Aufbau war dann wie folgt:\\
Auf die GPIO-Pins (siehe \autoref{fig:report_hardware_gpio1}) wird eine Platine(Header) gesteckt, welche die Transistorschaltung beinhaltet. Die Transistorschaltung schaltet die 5V Versorgung über einen Vorwiderstand den Optokoppler, welcher auf einer Platine in der Steckdose liegt. Die Verbindung wurde mit einem 2 poligen Draht, welcher auf beiden Seiten eine 3,5mm Mono Stecker montiert hat. Auf der Platine in der Steckdose ist nur der Optokoppler, der das Ein- bzw. Ausschalten der Steckdose vornimmt.\\
Für die Kommunikation mit dem Raspberry wird der GPIO Pin 4 (Header Pin 7) verwendet. Die 5V Spannung und die Masse für das Schalten des Optokopplers wird der Pin 2(5V) und der Pin 6(0V) des Raspberrys verwendet. Wichtig: Hier ist anzumerken, dass die Rasperrys Revision 1 und Revision 2 verschiedene Anordnung haben. Unsere Schaltung ist für Revision 2 entwickelt, es ist also nicht zu empfehlen, die Platine auf einem Revision 1 Raspberry zu verwenden, da er Schaden nehmen könnte.
\\
\begin{figure}[H]
\centering
\includegraphics[keepaspectratio=true, width=10cm]{images/rpi/picPins.png}
\caption[GPIO-Pins]{GPIO-Pins\\ \textbf{Quelle:} http://elinux.org/Rpi\_Low-level\_peripherals}
\label{fig:report_hardware_gpio1}
% source: http://elinux.org/Rpi_Low-level_peripherals#General_Purpose_Input.2FOutput_.28GPIO.29
\end{figure}
\begin{figure}[H]
\centering
\rule{1cm}{1cm}
\includegraphics[keepaspectratio=true, width=7cm]{images/rpi/gpio.png}
% source: http://elinux.org/Rpi_Low-level_peripherals#General_Purpose_Input.2FOutput_.28GPIO.29
\caption[GPIO-Pinbelegung]{GPIO-Pinbelegung\\ \textbf{Quelle:}
\label{fig:report_hardware_gpio2} http://elinux.org/Rpi\_Low-level\_peripherals}
\end{figure}
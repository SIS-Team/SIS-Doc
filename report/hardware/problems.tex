Während einer Entwicklung treten üblicherweise auch einige Probleme auf, an die nicht gedacht wurde, so auch in unserem Fall.\\
\subparagraph{Potentialtrennung}\label{sec:report_hardware_pot}
Das erste Problem das uns auffiel war das Problem der Potentialtrennung. Aufgrund dieses Problems kam es zu einem Kurzschluss, welcher den Raspberry zerstörte. Deshalb musste eine Potentialtrennung zwischen der Steckdose und dem Raspberry gemacht werden, damit keine hohen Spannungen auf den Raspberry übertragen werden und dadurch ein hoher Strom resultiert. Dies hat die Folge, dass der Raspberry zerstört wird. Die Lösung dieses Problem ist die Trennung mit Hilfe eines Optokopplers. Die Ausführung siehe Punkt ??? \\
\subparagraph{Spannung auf berührbaren Teilen}\label{sec:report_hardware_spannung}
Dieses Problem hat eigentlich wieder mit der Potentialtrennung zu tun. Wir lösten die Potentialtrennung auf dem Raspberry Header, das heißt das nicht potentialgetrennte Signal ist von der Steckdose über das Verbindungskabel zum Raspberry geführt worden. Dies ist jedoch nicht so dramatisch. Jedoch erzeugt die Steckdose ihr Potential für die Gleichspannungsversorgung immer fix. Das heißt die Steckdose nimmt immer, egal wie die Steckdose in der Steckdose steckt, zum Beispiel den rechten Pin der Steckdose.\\
Zu besseren Erklärung wird dies am folgendem Beispiel geschildert:\\
\begin{itemize}
	\item Nehmen wir an die Steckdose steckt mit der Position A in der Steckdose, somit hat die Schaltung als Masse den Neutralleiter, was dann, bezogen auf die Erdung, ergibt, dass die Masse gleich Erdung und die  5V gleich 5V+Erdungspotential ist. Somit ist alles in Ordnung.
	\item Nun nehmen wir an, dass die Steckdose mit Position B in der Steckdose steckt, was soviel wie eine Drehung um 180° bedeutet. Nun erzeugt die Schaltung wieder auf dem selben Prinzip die Versorgungsspannung. Nur mit dem Unterschied, dass wo zuvor der Neutralleiter war nun die Phase ist und umgekehrt. Dies hat zur Folge, dass die Schaltung die Versorgungsspannung auf Basis der Phase macht, also Masse gleich Phase gleich 230V und 5V gleich Phase+5V DC.
\end{itemize}
Somit lässt sich nun sagen, dass wir in unserem Fall, wenn die Steckdose \enquote{falsch in der Steckdose gesteckt ist}, 5V+230V über das Verbindungskabel auf den Respberry Header übertragen haben und somit bei Berührung des Headers einen 230V Stromschlag bekommen haben.\\
Die Lösung dafür war, dass wir die Potentialtrennung in die Steckdose verlegten und somit nur ein potentialfreies Signal über die Verbindungsleitung übertragen haben.
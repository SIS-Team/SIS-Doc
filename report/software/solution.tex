\subsubsection{Überlegungen}

Die Software zum regelmäßigen Anfragen am Server muss bereits beim Systemstart mitgestartet werden. Es bietet sich an, die Scripts über die Datei /etc/rc.local zu starten, da diese Datei relativ unkompliziert verwendet werden kann (\autoref{siehe lst:report_attachment_rc-local} Zeilen 20 und 22, Seite \pageref{lst:report_attachment_rc-local}).\\

\subsubsection{Darstellung}

Damit kein Tool für den automatischen Login installiert werden muss, wird der Default-Runlevel auf 2 gesetzt (Dies kann durch Modifikation des Schlüssels \enquote{initdefault} in der Datei /etc/inittab erreicht werden. Siehe \autoref{lst:report_attachment_inittab} Zeile 5, Seite \pageref{lst:report_attachment_inittab}.) und der X-Server manuell gestartet.\\
Das Starten des X-Servers wird durch das Script /usr/bin/sis-x initialisiert.

\lstinputlisting[style=custom, language=sh, caption={/usr/bin/sis-x}, label={lst:report_software_sis-x}]{sources/raspberry/sis-x.x}

Wie in \autoref{lst:report_software_sis-x} ersichtlich ist, wird überprüft, ob das Script als root gestartet wurde. Wenn dies der Fall sein sollte, dann startet das Script sich selbst mit Benutzerkennung \enquote{pi} neu, wechselt in das Heimverzeichnis des Benutzers (statisch eingetragen als /home/pi) und initialisiert mit dem Befehl \enquote{startx} einen neuen X-Server.\\
\\
X11 führt beim Starten die Datei .xinitrc im Heimverzeichnis des Benutzers (also /home/pi/.xinitrc) aus. Siehe dazu \autoref{lst:report_attachment_xinitrc}, Seite \pageref{lst:report_attachment_xinitrc}.\\
In dieser passiert folgendes:
\begin{itemize}
	\item Es wird verhindert, dass der Bildschirm abgedunkelt wird.
	\item Es wird ein Hintergrundbild gesetzt.
	\item Das Programm unclutter wird gestartet, welches den Mauszeiger automatisch ausblendet.
	\item Der Name des Monitors wird aus der Datei /etc/sis.conf ausgelesen.
	\item Nun wird die Session des Webbrowsers initialisiert (Chromium im App-Modus mit der Startseite http://sis.clients.htlinn.ac.at/monitors/?[Inhalt der /etc/sis.conf]).
	\item Eine Session des Display-Managers \enquote{blackbox} wird gestartet.
	\item Nach einer Verzögerungszeit von einer Sekunde wird das Hintergrundbild von Blackbox gesetzt.
	\item Nach weiteren 19 Sekunden wird mit dem Tool \enquote{xdotool} ein Tastendruck auf F11 simuliert, wodurch der Webbrowser in den Vollbild-Modus geschaltet wird.
\end{itemize}
\subsubsection{GPIO}
Für die Ansteuerung der GPIO Pins werden zwei Shell-Scripts verwendet.

\lstinputlisting[style=custom, language=sh, caption={/usr/bin/monitorOn}, label={lst:report_software_monitorsOn}]{sources/raspberry/monitorOn.x}

\lstinputlisting[style=custom, language=sh, caption={/usr/bin/monitorOff}, label={lst:report_software_monitorsOff}]{sources/raspberry/monitorOff.x}

Zuerst wird geprüft, ob der ausführende Benutzer root ist. Dies hat den Grund, dass nur root auf das GPIO-System schreiben darf.\\
Danach wird mit der Zeile 8 der GPIO-Pin 4 aktiviert. Der Pipe-Operator \enquote{2>\&1} dient dazu, dass etwaige Fehlermeldungen nicht ausgegeben werden (diese entstehen, wenn der Pin bereits als GPIO-Pin verwendet wird).\\
Nun (Zeile 9) wird der GPIO-Pin als Ausgang konfiguriert.\\
Also letztes wird nur noch der jeweilige Wert auf die Datei \enquote{value} geschrieben. Hierbei symbolisiert 1 das logische \textit{true}, und damit das Verhandensein der Steuerspannung (3.3 V) am Pin, und 0 das logische \textit{false}, also 0 V.

\subsubsection{Verteilung}

\paragraph{Clientseitig}
wird für die Verteilung der Befehle ein Script erstellt, das beim Systemstart mitgestartet wird (über /etc/rc.local) und danach regelmäßig Anfragen an den Server schickt.
Dieses Script wird unter /usr/bin/sis-monitors gespeichert.\\
Für den Inhalt siehe \autoref{lst:report_attachment_sis-monitors}, Seite \pageref{lst:report_attachment_sis-monitors} (Flußdiagramm siehe \autoref{fig:report_software_sis-monitors}).\\

\paragraph{Serverseitig} werden im API-Ordner der Monitore (/monitors/api/) zwei PHP-Scripts erstellt.\\
Das Script display.php gibt \enquote{true}, für eingeschaltenen Monitor, beziehungsweise \enquote{false}, für ausgeschaltenen Monitor, zurück (siehe \autoref{fig:report_software_sis-display} und \autoref{lst:report_attachment_display}).\\
Das Script restart.php dient dazu den X-Server des Raspberrys neuzustarten. Da dieses Verhalten nicht so häufig benötigt wird, wird dazu das Vorhandensein der Datei /tmp/restartx.ex geprüft (siehe \autoref{fig:report_software_sis-restart} und \autoref{lst:report_attachment_restart}). Für die Abarbeitung am Raspberry siehe \autoref{fig:report_software_sis-monitors}.
\\
\begin{figure}[H]
\centering
\resizebox{16cm}{!}{
	\fdot[scale=0.9]{images/flowcharts/sis-monitors}
}
\caption{Programm-Ablauf sis-monitors}
\label{fig:report_software_sis-monitors}
\end{figure}

\begin{figure}[H]
\centering
\resizebox{17cm}{!}{
	\fdot[scale=0.9]{images/flowcharts/display}
}
\caption{Programm-Ablauf display.php}
\label{fig:report_software_sis-display}
\end{figure}

\begin{figure}[H]
\centering
\fdot[scale=0.9]{images/flowcharts/restart}
\caption{Programm-Ablauf restart.php}
\label{fig:report_software_sis-restart}
\end{figure}
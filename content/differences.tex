Im Zuge des Projektes musste das Team feststellen, dass manche Teile des Pflichtenheftes nicht machbar waren.\\
Folgende Dinge wurden verändert:
\begin{itemize}
	\item Es erfolgt keine Sortierung der Ersatzlehrer nach Relevanz. Es wurde zwar eine Vorbereitung dafür eingebaut, jedoch ist dieses Modul aktuell nicht einsatzbereit.
	\item Beim Monitorsystem wurde zwar ein Modus für Videos vorgesehen - dieser ist auch auf normalen PCs lauffähig - allerdings mussten wir im Nachhinein feststellen, dass die Raspberry Pis nicht ohne weiteres in der Lage sind, HTML5-Videos abzuspielen.
	\item Bei den Eingaben gibt es bei kritischen Punkten (wie Lehrer oder Stunden) keine Möglichkeit, Einträge zu löschen. Dies hat den einfachen Grund, dass sonst das Risiko, dass ein Hacker (oder einfach nur jemand, der das Administrator-Passwort hat) das komplette System zum Kollaps bringt durch Löschen von Fremdschlüsseln einfach zu groß wäre.
	\item Die Benutzerwebsite wurde nicht als mobile Version implementiert, da es ja für die 3 wichtigsten Mobil-Betriebssysteme eine App gibt. Damit jedoch auch andere Benutzer unser System nutzen können, ist es möglich, am Handy eine JavaScript-freie Version der Website zu laden.
	\item Das Layout für die Website wurde nicht an das der HTL Homepage angepasst, sondern wurde vom externen Mitarbeiter Philipp Machac designed.
	\item Das Design der App wurde ebenfalls extra entwickelt.
	\item Es wurde keine Hilfe innerhalb der Website eingebaut, dafür ist auf der Website ein Link für den Download der PDF-Version der Benutzeranleitungen vorgesehen.
	\item Statt des Umweges mit dem Entfernen und Neu-Einfügen, wird für das Verschieben von Stunden ein eigener Menüpunkt vorgesehen.
	\item Es gibt keine Auflistung der Kollisionen (durch fehlende Lehrer etc.).
\end{itemize}
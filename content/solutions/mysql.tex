\subsection{MySQL}
Für das Datenbankmanagementsystem gibt es viele verschiedene denkbare Möglichkeiten. Durch den von uns bereitgestellten Server stellte sich die Frage, welches wir für unser Projekt verwenden jedoch nicht.
\subsubsection{Gewählte Lösung}
Als Datenbankmanagementsystem verwenden wir das kostenlos verfügbare MySQL. Dies ist unsere Wahl, da es einerseits am Server schon installiert war und es ist außerdem auch auf Linux Systemen verfügbar. Außerdem ist das PHP-Modul für MySQL am Server installiert, daher mussten keine weitere Module für die Datenbankanbindung installiert werden.
\subsubsection{Alternative Lösungen} 
\paragraph{MS SQL}
MS SQL ist ein von Microsoft entwickeltes relationales Datenbankmanagementsystem. Eines der großen Nachteile ist, dass es nur auf einem Windows Server lauffähig ist und man muss Lizenzgebühren zahlen.
\paragraph{Postgre SQL}
Postgre SQL ist eine freie und objektrelationales Datenbankmanagementsystem, welches auf Windows, Linux und Unix-Distributionen lauffähig ist. Postgre SQL ist nicht komplett nach der SQL Konformität entwickelt, das heißt, dass die meisten der SQL Befehle funktionieren und deren Dienst laut ihrer Definition erweisen, jedoch nicht alle. Zur Bedienung der Datenbank gibt es ein Komandozeilentool mit dem Namen psql und als grafische Verwaltung gibt es einige kommerzielle Tools und auch freie Tools wie phpPgAdmin.
\paragraph{Oracle Database}
Oracle Database ist ein Datenbankmanagementsystem, welches in einer relationalen und objektrelationalen Version verfügbar ist. Oracles Produkt ist neben MS SQL das am weit verbreitetste Datenbanksystem. Es gibt verschiedene Versionen, für schulische Zwecke können einige Versionen kostenfrei erworben werden, diese sind jedoch eingeschränkt. Für den kommerziellen Gebrauch muss jedoch eine Lizenz erworben werden. Ein interessanter Fakt ist, dass das Produkt von Oracle auf bis zu $ 40 * 2^{60} $ Byte an Datenbank zugreifen kann. Es gibt einige Tools die von Oracle zur Verfügung gestellt werden, jedoch gibt es auch einige Tools von anderen Herstellern, die das Verwalten von Oracle Datenbanken erlaubt. Oracle stellt auch eigene Hardware für ihre Datenbank zur Verfügung.
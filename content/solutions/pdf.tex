\subsection{PDF-Generierung}
Die PDF-Generierung wird mittels FPDF durchgeführt.\\
\\
Der Vorteil in dieser Variante gegenüber anderen PDF-Lösungen liegt darin, dass auf dem Server nichts installiert werden muss. Es muss nur das Archiv, welches auf der Website von FPDF (\href{www.fpdf.org}{www.fpdf.org}) downloadbar ist, auf dem Server entpackt werden. Von diesem Verzeichniss werden jedoch nur die Datei \enquote{fpdf.php} und der Ordner mit den Font-Styles benötigt. Die restlichen Dateien wurden gelöscht.\\
Zudem bietet eine Lösung mittels generiertem PDF auch den Vorteil, dass der Ausdruck ohne großen Aufwand auf einem Rechner gespeichert werden kann.
\\
Ein Nachteil er Verwendung von FPDF ist jedoch, dass das Design aufgrund der Zellenstruktur nur schwer zu Ändern bzw. Erstellen ist. Besonders die Kopf und Fußzeile sind nicht leicht zu verändern.\\\\
Eine andere Möglichkeit eine Druckausgabe zu erzeugen, wäre, wie im Vorgängerprojekt verwendet wurde, eine ausdruckbare Website. In diesem Fall wäre der Ausdruck jedoch nicht sofort als PDF erstellt worden und hätte um ein PDF zu erstellen mit einem PDF-Drucker ausgedruckt werden müssen.
% Wie wird das PDF generiert? Warum haben wir welche Lib verwendet? Alternativen? Vor-, Nachteile?

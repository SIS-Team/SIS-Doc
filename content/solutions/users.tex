\subsection{Benutzermanagement}
\label{sec:content_solutions_users}
In der Schule sind im Wesentlichen 4 Berechtigungsstufen vorhanden.
\begin{itemize}
	\item Schüler
	\item Lehrer
	\item Abteilungs-Vorstände
	\item Administratoren
\end{itemize}
Diese Gliederung wird 1:1 in SIS übertragen, allerdings wird zusätzlich eine Gruppe für die \enquote{Newsbeauftragten}, welche gegenüber den Schülern/Lehrern zusätzlich die Berechtigung haben, den Abteilungsvorständen News vorzuschlagen, vorgesehen.\\
Da die Benutzer-Authentifizierung über den LDAP-Server läuft, ist es naheliegend, dass auch die Autorisierung über LDAP laufen könnte.\\
Weil alle Lehrer-Objekte automatisch im eDirectory-Container ou=LEHRER,o=HTLinn und die Schüler-Objekte automatisch im eDirectory-Container ou=STUDENTS,o=HTLinn liegen, lässt sich anhand des DN des Benutzers auslesen, ob es sich um einen Lehrer oder um einen Schüler handelt.\\
\subsubsection{eDirectory-Gruppen}
Für alle weiteren logischen Gruppen werden vom eDirectory-Administrator auch dementsprechende eDirectory-Gruppen erstellt:
\begin{itemize}
	\item cn=SIS-SuperUser,ou=SIS,o=HTL1
	\item cn=SIS-News,ou=SIS,o=HTL1
	\item cn=SIS-Admin-E,ou=SIS,o=HTL1
	\item cn=SIS-Admin-M,ou=SIS,o=HTL1
	\item cn=SIS-Admin-N,ou=SIS,o=HTL1
	\item cn=SIS-Admin-W,ou=SIS,o=HTL1
\end{itemize}
Es ist prinzipiell möglich, dass ein Benutzer Mitglied mehrerer eDirectory-Gruppen ist.\\
\subsubsection{Verwendung}
Meldet sich ein Benutzer an, so werden seine Berechtigungen (sprich Gruppen-Zugehörigkeiten) am LDAP-Server abgefragt und in der PHP-Session gespeichert. Auch, ob es sich um einen Lehrer handelt, wird in der Session gespeichert. Dies hat den Vorteil, dass nur einmal die Berechtigungen abgefragt werden müssen.\\
Jedes PHP-Script kann die Berechtigungen über die PHP-Session abfragen und mehr oder weniger autonom entscheiden, welche Berechtigungen der Benutzer erhält.
\chapter{Pflichtenheft}

\section{Funktionale Anforderungen}

\subsection{Definitionen}
\begin{description}[style=nextline]
	\item[angepasster Stundenplan]
		Stundenplan mit eingearbeiteten Supplierungen 
	\item[tabellarischer Supplierplan]
		Auflistung aller Supplierungen, Ausfälle, etc
	\item[Relevanz bei Ersatzlehrern]
		Ist der Lehrer an diesem Tag nicht in der Schule, so ist er als Supplierlehrer nicht erste Wahl (kursiv oder grau hinterlegt darstellen). Ist er an diesem Tag in der Schule, hat jedoch Unterricht, so ist er nicht als Supplierlehrer einsetzbar, ist er jedoch als Zweitlehrer im selben Unterricht mit dem Absenzlehrer, dann kann er als \enquote{Klasse alleine} eingeteilt werden, etc.
\end{description}

\subsection{Supplierungssystem}
Es soll ein System entwickelt werden, dass die Stundenpläne und Supplierungen in digitaler Form speichert. Dazu soll eine Ein- und Ausgabe der Daten über eine Website und eine App zur Verfügung gestellt werden (siehe Punkt Ausgabe). Es weiteren wird ein Formular generiert (PDF), das ausgedruckt werden kann.\\
Die Eingabe über die App ist nur eingeschränkt und wenn zeitlich möglich. Die Supplierungen und Stundenpläne werden vom Administrator (AV, WL, ...) eingegeben (siehe Punkt Eingabe).

\subsection{News}
News sollen vom Administrator (AV,WL, ...) eingegeben werden (abrufbar über die Website bzw. App).

\subsection{Monitorsystem}
Thin-Clients (z.B.: Raspberry Pi) mit Monitoren sollen mit Daten versorgt werden. Dazu soll nur ein HTML5 kompatibler Browser benötigt werden. Der dementsprechende HTML-­Code soll möglichst auflösungskompatibel sein.
Es soll möglich sein das, auf den Thin-Clients, dargestellte individuell über die Website zu konfigurieren.\\

Folgende Möglichkeiten:
\begin{itemize}
	\item 
		angepasster Stundenplan des nächstgelegenen Raumes
	\item
		tabellarischer Supplierplan der Abteilung (mit Informationen bzgl.: Magazin und den News)
	\item
		Bild als JPG, PNG oder GIF (Upload über die Website)
	\item
		Video im MP4-Container (Upload über die Website)
	\item
		Uhr
\end{itemize}

\subsection{Authentifizierung}
Authentifizierung erfolgt für die Schüler und Lehrer via LDAP, gilt auch für Monitore (diese müssen sich als Monitore identifizieren). Ohne erfolgreichen Login sind keine Informationen abrufbar.

\subsection{Eingabe}
Administratoren und AVs dürfen Eingaben tätigen. Damit einfache Eingaben auch delegiert werden können muss ein Berechtigungssystem hinterlegt werden.

\begin{description}[style=nextline]
	\item[Lehrer]
		Name, Initialen, Abteilung\\
		Buttons zum Hinzufügen, Editieren und Löschen (LDAP)
	\item[Klassen]
		Name, KV (als Dropdown-Menü), Abteilung (als Dropdown-Menü), Raumbelegung
	\item[Räume]
		Bezeichnung, Abteilung
	\item[Fächer]
		Bezeichnung (Kürzel und Langname)
	\item[Stunden(-pläne)]
		Fach, Lehrer (Dropdown-Menü; weitere Felder erscheinen bei der Auswahl), Dauer, Raum (Dropdown-Menü)\\
		\\
		Auswahl der Klasse über ein Menü. Stundenplan aus \enquote{Klassen-­Sich}t. Liste der Wochentage und Buttons zum Hinzufügen, Platzieren, Editieren und Löschen von Stunden im Stundenplan.
	\item[Supplierungen]
		Drei Eingaben:
		\begin{description}[style=nextline]
			\item[fehlende Lehrer]
				Lehrer (DropDown-Menü), von-bis, Grund
			\item[fehlende Klassen]
				Klasse (DropDown-Menü), von-bis, Grund
			\item[Supplierungen]
				Stunde (Dropdown­-Menü), Klasse (Dropdown-Menü), Ausblenden (Check-Box; wenn gesetzt, wird diese Stunde in den angepassten Stundenplänen nicht angezeigt), Supplierlehrer (Dropdown-Menü; zeigt die Lehrer sortiert und markiert nach Relevanz), Kommentar (Hier wird eingetragen z.B.: \enquote{Mitbetreuung}, \enquote{Stillbeschäftigung}, \enquote{entfällt} etc.), Bestätigen (Check-Box; Eintrag ist erst wirksam, wenn gesetzt)\\
			\\
			Ein Supplierlehrer muss bei Mitbetreuung nicht angegeben werden, da alle anderen Lehrkräfte dieser Stunde, sowieso mit dieser verknüpft sind.\\
			\\
			Verschobene Stunden werden als 2 Einträge, einmal \enquote{ausgefallen} (mit dem \enquote{ausblenden}-­Button) und einmal \enquote{neu eingefügt} (gekennzeichnet über Kommentar) eingegeben.\\
			\\
			(ev. falls noch Zeit: Wenn ein fehlender Lehrer eingetragen wurde, so werden automatisch alle \enquote{Kollisionen} angezeigt.)
		\end{description}
	\item[News]
		Name, Beschreibung, von-bis, Abteilung (Dropdown-Menü; auch mit Auswahl für die ganze Schule), die News werden nach Ablauf (Bis-Datum) nicht mehr angezeigt, aber nicht gelöscht.
	\item[Monitore]
		Modus (Auswahlliste, siehe Punkt Monitorsystem), falls benötigt: Datei (Upload für Bild, Video)\\
		\\
		Die Monitore melden sich selbst in der DB an, so ist kein Hinzufügen von Monitoren nötig. \\
		Allerdings: Möglichkeit zum Sperren von Einträgen, sollte sich ein Monitor verändern.\\
		\\
		Über Check-Boxen wählt man alle oder einzelne Monitore aus, bei denen man die Konfiguration ändern will. Buttons für alle, keinen und einzelne auswählen.
	 \item[Ausgabe]
	 	Hier gibt es 2 verschiedene Möglichkeiten:
	 	\begin{description}[style=nextline]
	 		\item[Benutzer-Website/App]
	 			Nach Login:\\
	 			Für Schüler und Lehrer wird ein Klassen-/Lehrerspezifisch angepasster Stundenplan generiert. Über einen Button auf der Startseite kann die Anzeige-Art verändert werden.
	 		\item[Monitore]
	 			siehe Punkt Monitorsystem
	 	\end{description}
	\item[App]
		Es soll eine App für Android, Windows Phone und iOS erstellt werden, die die gleichen Funktionen bietet wie die Standard-Benutzer-Website (keine administrativen Funktionen).\\
		\\
		Zusätzlich soll die Benutzer-Website (aufgrund der Kompatibilität zu anderen Mobil-Betriebssystemen) auch als mobile Website implementiert werden.
	 \item[Formular]
	 	Das Formular für die Übertragung der Supplierungen in das Abrechnungssystem wird nach derzeitiger Vorlage generiert. Ein weiteres Formular wäre sinnvoll: Die Auflistung nach fehlendem Lehrer, damit man einen Überblick erhält:\\
	 	\\
	 	\textit{Bsp:}\\
	 	YH fehlend:
\begin{tabbing}
1.6. \= 1. Std. \= TKHF \= 1aHEL \hspace{2em} \= Nz\\
 \> 2. Std. \> TKHF \> 2aHEL \> MT\\
2.6. \> 3. Std. \> LA1 \> 4aHEL \> XY
\end{tabbing}
		...
	\item[Layout]
		Die Eingabeseite/Eingabenmasken
		 sollen übersichtlich und einfach zu bedienen sein. Das Layout wird der neuen HTL Homepage angepasst (Corporate Design) - als Grundlage dient das FTKL Projekt (Machac, Handle, Wucherer).\\
		\begin{description}[style=nextline]
			\item[Stundenplandesign]
				Als Vorgabe dienen die derzeitigen Raumbeschriftungen der Werkstätten – das Layout wird wieder an das neue Corporate Design angepasst.
			\item[App­Design]
				siehe Corporate Design
			\item[Stundenplaneingabe]
				Am Schuljahresanfang wird der Stundeplan der Abteilung händisch ins SIS übertragen. Die Grundlage für die Eingabe ist der Klassenstundenplan. Da Lehrer auch in anderen Abteilungen eingeteilt werden können, muss es für den jeweiligen Administrator möglich sein, auch diese Stunden einzugeben. Die Eingabemaske soll dem Wochenstundenplan angepasst sein (Stunde (1-16) Fach, Klasse, Raum).
			\item[Dokumentation]
				Die Dokumentation wird lt. Vorlage (Mail von Prof. Stecher) ausgeführt. Es sind Bedienungs- und Serviceanleitungen zu erstellen. Mit diesen Unterlagen muss eine Weiterentwicklung (für andere Diplomanten) und eine Servicierung durch das Lehrpersonal gewährleistet sein. Der Sourcecode ist sauber zu dokumentieren. Eine Hilfe im Programm im HTML Format ist zu erstellen.\\
				\\
				Ein Projekttagebuch ist zu führen (Beginn des Tasks/Sprints; Zeit und Task; Unterbrechungen; Status)\\
				\\
				Code im Code dokumentieren: doxygen/javadoc
		\end{description}
	\item[Uhranzeige]
		Auf jedem Monitor ist eine Zeitanzeige zu sehen und diese wird dem Design der Anzeigeseite angepasst (Corporate Design).
\end{description}

\section{Schnittstellen}
Es wurde zwar eine Software-Schnittstelle zur verwendeten Schul-Management-Software Untis angedacht, diese Idee wurde aber verworfen, da die Sinnhaftigkeit aufgrund des kommenden Umstiegs der Schule auf eine neue Version in Frage gestellt wird.

\section{Abnahmekriterien}
% ist in echt gelaufen
% getestet durch marth, huber, stecher
% beta betrieb in bestimmten klassen

Der Service läuft jetzt schon seit 2 Monaten im Probebetrieb, der sich auf die N-Abteilung begrenzt, ohne wesentliche Probleme. Es bestanden gewisse Startschwierigkeiten, welche aber in kurzer Zeit behoben werden konnten. Das Monitorsystem läuft ohne grobe Ausfälle, gegen die wir Maßnahmen treffen könnten.\\
Die App ist nach einigen kleineren Schwierigkeiten nun seit einer Woche in allen App Stores für Windows Phone, iOS und Android vorhanden und kann verwendet werden.\\
Das Webinterface und die Eingaben für die Administratoren wurden in mehreren Absprachen mit den zuständigen Personen optimiert und können jetzt ohne weitere Unterstützung seitens des SIS-Teams seit Wochen verwendet werden.\\
Die Alltagstauglichkeit der Eingaben wurden über Wochen hinweg von Herrn OSTR. Prof. Mag. Dr. HUBER Josef, AV OStR Prof. DI MARTH Walter und Herrn WL. FOL. Dipl.-Päd. DI STECHER Helmut getestet.\\
Es wurden keine konkreten Abnahmekriterien vorgegeben, allerdings wurde, wie bereits erwähnt, das Projekt ständig kontrolliert, überprüft und getestet. Da nach nun 2 Monaten Laufzeit keine gröberen Probleme aufgetaucht sind, sehen wir das Projekt als abgeschlossen.


\section{Qualitätsstandards}
Um die Qualität während der Entwicklung des Projekts einigermaßen zu erhalten, wurde der Code fortlaufend ausgetauscht. Dazu wurde GitHub verwendet. Durch den ständigen Austausch des Codes wird vermieden, dass Teile des Programmes doppelt entwickelt werden und dass bei Problemen andere Mitarbeiter sofort Einsicht in den Code haben.\\
Weiters wurde versucht bei der Programmierung auf einen einigermaßen einheitlichen Programmierstil zu achten, deshalb wurde auch grundsätzlich auf Englisch programmiert und die Dateien wurden alle englisch benannt. (Ausnahme: lokale Dateien der App)\\
Bei Fremd-Software wurde auf Seriosität geachtet. Zum Beispiel PhoneGap, das Framework zum Erstellen der App, wird von Adobe-Systems zur Verfügung gestellt. Dadurch ist bestätigt, dass es sich bei PhoneGap um ein seriöses Produkt handelt.\\
Um auch eine gewisse Sicherheit zu gewährleisten, werden alle Verbindungen über HTTPS, also verschlüsselt, aufgebaut. Dadurch werden die Daten sicher an den Server übertragen.\\
Auch aus Gründen der Sicherheit wurde auf die MySQL-Verwaltungssoftware PhpMyAdmin verzichtet. Diese könnte eine potenzielle Sicherheitslücke darstellen.\\
\\
Ein weiteres Qualitätsmerkmal stellt die Bedienungsfreundlichkeit dar. Da dieses Projekt davon lebt, dass es von möglichst vielen Personen genutzt wird, ist es wichtig, dass der Nutzer die Software gerne nutzt.\\
Darum muss darauf geachtet werden, dass die Bedienung für den Benutzer möglichst einfach und unkompliziert ist.
Aber auch die Bedienung für den Administrator muss trotz der vielen Einstellungen und Menüpunkte übersichtlich bleiben.\\
 


\section{Prozessmodell}
% abgehört mit abgabe
% "gearbeitet" mit sprint-pläne
% wirklicher zeitplan war nicht möglich
%  -> weil fehlende erfahrung.
% immer wieder rücksprachen mit stecker

\section[Abweichungen]{Abweichungen von der Aufgabenstellung}
Im Zuge des Projektes musste das Team feststellen, dass manche Teile des Pflichtenheftes nicht machbar waren.\\
Folgende Dinge wurden verändert:
\begin{itemize}
	\item Es erfolgt keine Sortierung der Ersatzlehrer nach Relevanz. Es wurde zwar eine Vorbereitung dafür eingebaut, jedoch ist dieses Modul aktuell nicht einsatzbereit. Die Arbeiten dafür wurden eingestellt, da der Abteilungsvorstand normalerweise schon im vorhinein weiß, wen er eintragen will.
	\item Beim Monitorsystem wurde zwar ein Modus für Videos vorgesehen - dieser ist auch auf normalen PCs lauffähig - allerdings mussten wir im Nachhinein feststellen, dass die Raspberry Pis nicht ohne weiteres in der Lage sind, HTML5-Videos abzuspielen.
	\item Auch wurde die Information über das Elektronik-Magazin auf den Monitoren weggelassen.
	\item Bei den Eingaben gibt es bei kritischen Punkten (wie Lehrer oder Stunden) keine Möglichkeit, Einträge zu löschen. Dies hat den einfachen Grund, dass sonst das Risiko, dass ein Hacker (oder einfach nur jemand, der das Administrator-Passwort hat) das komplette System zum Kollaps bringt durch Löschen von Fremdschlüsseln einfach zu groß wäre.
	\item Die Benutzerwebsite wurde nicht als mobile Version implementiert, da es ja für die 3 wichtigsten Mobil-Betriebssysteme eine App gibt. Damit jedoch auch andere Benutzer unser System nutzen können, ist es möglich, am Handy eine JavaScript-freie Version der Website zu laden.
	\item Das Layout für die Website wurde nicht an das der HTL Homepage angepasst, sondern wurde vom externen Mitarbeiter Philipp Machac designed.
	\item Das Design der App wurde ebenfalls extra entwickelt.
	\item Es wurde keine Hilfe innerhalb der Website eingebaut, dafür ist auf der Website ein Link für den Download der PDF-Version der Benutzeranleitungen vorgesehen.
	\item Statt des Umweges mit dem Entfernen und Neu-Einfügen, wird für das Verschieben von Stunden ein eigener Menüpunkt vorgesehen.
	\item Es gibt keine Auflistung der Kollisionen (durch fehlende Lehrer etc.).
\end{itemize}

\chapter[Zusammenfassung]{Zusammenfassung des Projektergebnisses}

\section{Kurzbeschreibung}
Im Zuge dieser Diplomarbeit wurde SIS (School Information Service) entwickelt. \\
Dieses System stellt über ein Webinterface den Schülern und Lehrern den Stundenplan und Supplierplan, sowie aktuelle News zur Verfügung. Zusätzlich wird ein Stundenplan generiert, in dem bereits die supplierten Stunden hervorgehoben werden.
Für diese Funktionalität wird auch eine App für die Mobilbetriebssysteme iOS, Android und Windows Phone zur Verfügung gestellt.\\
Darüber hinaus werden auf den Monitoren, die vor den Werkstätten und manchen Klassenräumen positioniert sind, je nach Einstellung, die aktuellen schulrelevanten Neuigkeiten, die Supplierpläne der Abteilung, der Stundenplan des nächstgelegenen Raumes oder benutzerdefinierte Bilder angezeigt.\\
Die News können von den Administratoren der Abteilungen eingetragen, sowie von den News-Beauftragten der Klassen vorgeschlagen werden.\\
\\
Das System wurde für alle 4 Abteilungen ausgelegt, aber die Nutzung nur in der Elektronik-Abteilung forciert.

\section{Projektergebnis}
Als ein Projektteam, welches versucht hat ein digitales School Information Service zu implementiert, können wir das erste Mal behaupten, dass unser System verwendet wird und eine mögliche Zukunft hat. (siehe \gref{sec:content_draft_log_erkenn})\\
Die App ist in den drei großen App-Stores (Google Play Store, Apple iTunes Store und Microsoft Store) vertreten.\\
232 von 449 Schülern, sowie 19 Lehrer, haben sich bereits mindestens einmal an unserem Service angemeldet. \\
\\
Die Rückmeldungen der Schüler und Lehrer sind nach Beseitigung anfänglicher Probleme vorwiegend positiv.

\subsection{Abschätzung des Traffics}
Der durch unser Projekt verursachte Traffic wird auf ungefähr 700 MB pro Tag geschätzt. Im Folgenden wird die Berechnung ausführlich beschrieben.\\

\paragraph{Messwerte\\}
Die für die Abschätzung herangezogenen Messwerte stammen vom 18.3.2014 9:15 Uhr - 18.3.2014 14:15.\\
Dabei wurden folgende Werte beobachtet:

\begin{itemize}
	\item ca. 600 App-Aufrufe\\
	je $ \approx  $ 10 kB (Stark abhängig von der aufgerufenen Seite. Exemplarisch: Stundenplan)
	\item ca.  800 Website-Aufrufe\\
	je $ \approx  $ 130 kB (Stark abhängig von der aufgerufenen Seite. Exemplarisch: Stundenplan-Formular -> relativ groß)
	\item ca 34000 Monitor-Aufrufe\\
	je $ \approx  $   1 kB (Stark abhängig von Änderungen. Geschätzter Wert; maximal zirka 10 kB, minimal ~ 50 B)
\end{itemize}

\paragraph{Berechnung\\}

Im Folgenden werden der Traffic der Monitore, App-Aufrufe und Website-Aufrufe zusammengezählt um den gesamten Traffic zu erhalten, welcher vom Webserver verursacht wird.\\
\\
Aufrufe aus der App: $ 600 \cdot 10 kB = 6MB $\\
Aufrufe aus dem Web: $ 800 \cdot 130 kB = 104 MB $\\
Aufrufe der Monitore: $ 3400 \cdot 1kB = 3,4MB $\\
\\
Daraus ergibt sich eine Summe von 113,4MB innerhalb von 4 Stunden, dies auf 24h Stunden hochgerechnet ergibt einen Traffic von 680 MB.\\

\paragraph{Interpretation\\}
Dieses Ergebnis ist sehr stark geschätzt, da für unsere Berechnung ein Ausschnitt von 4 Stunden an einem spezifischen Tag gewählt wurde.\\
Dieses Ergebnis ist zumindest ein Anhaltspunkt, falls eine Begrenzung des Traffics durchgeführt werden soll oder ähnliche Maßnahmen getroffen werden sollen.
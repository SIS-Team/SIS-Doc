\chapter[Zusammenfassung]{Zusammenfassung des Projektergebnisses}

\section{Kurzbeschreibung}
Im Zuge dieser Diplomarbeit wurde SIS (School Information Service) entwickelt. \\
Dieses System stellt über ein Webinterface den Schülern und Lehrer den Stundenplan und Supplierplan, sowie aktuelle News zur Verfügung. Zusätzlich wird ein Stundenplan generiert, in dem bereits die supplierten Stunden hervorgehoben werden.
Für diese Funktionalität wird auch eine App für die Mobilbetriebssysteme iOS, Android und Windows Phone zur Verfügung gestellt.\\
Darüber hinaus werden auf den Monitoren, die vor den Werkstätten und manchen Klassenräumen positioniert sind, je nach Einstellung, die aktuellen schulrelaventen Neuigkeiten, die Supplierpläne der Abteilung, der Stundenplan des nächst-gelegenen Raumes, oder benutzerdefinierte Bilder angezeigt.\\
Die News können von den Administratoren der Abteilungen eingetragen, sowie von den News-Beauftragten der Klassen vorgeschlagen werden.\\
\\
Das System wurde für alle 4 Abteilungen ausgelegt, aber die Nutzung nur in der Elektronik-Abteilung forciert.

\section{Projektergebnis}
Die App ist in den drei großen App-Stores (Google Play Store, Apple iTunes Store und Microsoft Store) vertreten.\\
Von den 449 Schülern der Elektronikabteilung haben sich 232, sowie 19 Lehrer, bereits mindestens einmal angemeldet. \\
Der durch unsere Projekt verursachte Traffic wird auf ungefähr 850 MB pro Tag geschätzt (genauer Rechenweg siehe \gref{sec:traffic}).\\
\\
Die Rückmeldungen der Schüler und Lehrer sind nach dem Beseitigen anfänglicher Probleme vorwiegend positiv.
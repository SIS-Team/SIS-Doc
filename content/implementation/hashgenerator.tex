\subsection{Token-Generator}
\label{sec:content_imple_hashgenerator}
Bei der Implementierung der unter \gref{sec:content_draft_token} beschriebenen Klasse ist das erste Problem, das auftritt, dass es nur eine PHP-Session gibt, und somit auch nur ein benutzerspezifisches Daten-Array. Dies wird dann zu einem Problem, wenn für mehrere Formulare gleichzeitig die Tokens gespeichert werden müssen.\\
Zur Lösung dieses Problems werden sogenannte virtuelle Sessions verwendet.

\subsubsection{Virtuelle Sessions}
Die virtuellen Sessions sind Unterteilungen unterhalb der PHP-Session. Sie werden mit Hilfe der Klasse \texttt{vSession} verwaltet.\\
Die Klasse hat folgende Member:
\begin{description}[style=nextline]
	\item[\texttt{private \$name}] beinhaltet den Namen der virtuellen Session für das Objekt verwendet werden soll.
	\item[\texttt{public function init()}] initialisiert virtuelle Session, für welche das Objekt erstellt wurde. Sollte die Session bereits existieren, wird eine Exception geworfen (siehe \autoref{lst:content_imple_vSession_init}).\\
		\textit{Achtung:} Es wird die virtuelle Session initialisiert, nicht das Objekt.
	\item[\texttt{public function exists()}] prüft, ob die virtuelle Session, für welche das Objekt erstellt wurde und gibt einen bool'schen Wert zurück.
	\item[\texttt{public function destroy()}] löscht die Session, für welche das Objekt erstellt wurde.
	\item[\texttt{public function destroyAll()}] löscht alle virtuellen Sessions (genauer: löscht die Referenz auf das Array der virtuellen Sessions).
	\item[\texttt{public function setAttribute(\$name, \$value)}] setzt die durch \texttt{\$name} identifizierte Eigenschaft der virtuellen Session auf den Wert \texttt{\$value} (siehe \autoref{lst:content_imple_vSession_set}). Alte Werte werden überschrieben. Wenn mehr virtuelle Sessions existieren als erlaubt, wird \texttt{\$this->deleteOldest()} aufgerufen.
	\item[\texttt{public function getAttribute(\$name)}] gibt den Wert der durch \texttt{\$name} identifizierten Eigenschaft der virtuellen Session zurück.\\
		\textit{Achtung:} Es wird nicht überprüft, ob die Eigenschaft existiert.
	\item[\texttt{public function attributeIsset(\$name)}] gibt in Form eines bool'schen Wertes zurück, ob die durch \texttt{\$name} identifizierte Eigenschaft der virtuellen Session existiert.
	\item[\texttt{public function deleteOldest()}] löscht die älteste virtuelle Session (siehe \autoref{lst:content_imple_vSession_oldest}).
\end{description}
Zu erwähnen ist noch, dass die Constructor-Methode dieser Klasse als Parameter den Namen der Session erwartet (siehe \autoref{lst:content_imple_vSession_construct}).\\
Für alle virtuellen Sessions gibt es zusätzlich noch die Eigenschaften \texttt{created} und \texttt{modified}, welche beide Timestamps enthalten. Diese sind aber nicht zur Modifikation durch den Benutzer bestimmt, sondern dienen der vSession-Klasse als Anhaltspunkt zur Bestimmung alter Elemente, die vermutlich nicht mehr gebraucht werden. Initialisiert werden diese beiden Eigenschaften durch die \texttt{init()}-Methode (siehe \autoref{lst:content_imple_vSession_init}, Zeile 46, 47). Das \texttt{modified}-Feld wird dazu noch von der \texttt{setAttribute(\$name, \$value)}-Methode modifiziert (siehe \autoref{lst:content_imple_vSession_set}, Zeile 88). Zur Bestimmung, ob die Session noch benutzt wird, wird logischerweise der \texttt{modified}-Wert verwendet (siehe \autoref{lst:content_imple_vSession_oldest}, Zeile 123).\\
Die in den folgenden Codes sichtbare Konstante \texttt{MAX\_VSESSIONS} wird am Beginn der Datei \enquote{/modules/general/VirtualSessionManager.php} definiert. Der Standardwert ist 15.\\
\lstinputlisting[style=custom, language=php, caption={/modules/general/VirtualSessionManager.php; Constructor; Zeilen 29-31}, label={lst:content_imple_vSession_construct}, firstline=29, lastline=31, firstnumber=29]{sources/token/VirtualSessionManager.php}
\lstinputlisting[style=custom, language=php, caption={/modules/general/VirtualSessionManager.php; \texttt{init()}; Zeilen 38-51}, label={lst:content_imple_vSession_init}, firstline=38, lastline=51, firstnumber=38]{sources/token/VirtualSessionManager.php}
\lstinputlisting[style=custom, language=php, caption={/modules/general/VirtualSessionManager.php; \texttt{setAttribute()}; Zeilen 84-90}, label={lst:content_imple_vSession_set}, firstline=84, lastline=90, firstnumber=84]{sources/token/VirtualSessionManager.php}
\lstinputlisting[style=custom, language=php, caption={/modules/general/VirtualSessionManager.php; \texttt{deleteOldest()}; Zeilen 117-130}, label={lst:content_imple_vSession_oldest}, firstline=117, lastline=130, firstnumber=117]{sources/token/VirtualSessionManager.php}
Aus Platzgründen werden hier nur die wichtigsten Codeteile vorgestellt. Für den vollständigen Code siehe \gref{sec:content_imle_source}.

\subsubsection{HashGenerator}
Das direkte Arbeiten mit der Klasse für die virtuellen Sessions ist zwar möglich, aber nicht sehr komfortabel. Deshalb wurde eine weitere Klasse erstellt, die das Erstellen und Auswerten der Site-Tokens erleichtern soll. Diese greift dann auch auf die Methoden von \texttt{vSession} zu, um ein gewisses Maß an Abstraktion zu schaffen.\\
Die Klasse \texttt{HashGenerator} hat folgende Member:
\begin{description}[style=nextline]
	\item[\texttt{private \$name}] ist der vom Benutzer gewählte Name für das Formular.
	\item[\texttt{private \$sessionName}] wird beim Erstellen der Hashes aus dem gewählten Namen generiert. Dient dazu, dass nicht zwei Formulare mit dem selben Namen sich gegenseitig überschreiben.
	\item[\texttt{private \$id}] ist eine vom Benutzer angegebenen ID. Diese ID \textit{muss} für jedes Formular verschieden sein.
	\item[\texttt{private \$hash}] enthält, wie der Name schon sagt, den generierten Hash.
	\item[\texttt{private \$virtualSession}] dient als Speicher für die Instanz von \texttt{vSession}, die für dieses Objekt verwendet wird.
	\item[\texttt{public function generate()}] generiert die Hashes.
	\item[\texttt{public function printForm()}] gibt die versteckten Inputs für das Formular aus.
	\item[\texttt{public function check()}] überprüft die Hashes und wirft bei Fehler eine Exception. Der Fehlertyp steht in der Message der Exception.
\end{description}
Der Konstruktor erwartet als Parameter den Namen des Formulars und die ID. Es werden darin die jeweiligen privaten Felder gesetzt. Der Formular-Name wird zusätzlich als Initialisierungswert in das Feld \texttt{\$sessioName} geschrieben.
\lstinputlisting[style=custom, language=php, caption={/modules/form/HashGenerator.php; \texttt{generate()}; Zeilen 24-42}, label={lst:content_imple_hash_generate}, firstline=24, lastline=42, firstnumber=24]{sources/token/HashGenerator.php}
Wie in \autoref{lst:content_imple_hash_generate} zu sehen ist, wird zuerst mit dem bereits gesetzten Standard-Session-Namen eine \texttt{vSession}-Instanz erstellt. Der Name der Session wird solange modifiziert, bis ein unbenutzter Session-Name gefunden wurde (Theoretisch hat die \texttt{for}-Schleife in Zeile 28 maximal 14 Durchläufe, da es standardmäßig nur 15 virtuelle Sessions geben darf.).\\
Wurde ein passender Session-Name gefunden, so wird die virtuelle Session initialisiert, sprich, es wird der interne Aufbau der virtuellen Session hergestellt.\\
Nun wird der Hash berechnet. In diesem Fall wird der \texttt{sha256}-Hash vom Zahlenwert, des aktuellen Timestamps multipliziert mit einer zufälligen Zahl, verwendet. Diese Berechnung kann theoretisch beliebig verändert werden, es sollte nur darauf geachtet werden, dass die resultierende Zeichenkette nicht in vertretbarer Zeit erraten werden kann.\\
Nun werden nur noch die ID des Formulars, der Hashwert und die Zeit in der virtuellen Session gespeichert.
\lstinputlisting[style=custom, language=php, caption={/modules/form/HashGenerator.php; \texttt{printForm()}; Zeilen 43-46}, label={lst:content_imple_hash_print}, firstline=43, lastline=46, firstnumber=43]{sources/token/HashGenerator.php}
Die Ausgabe der versteckten Formularelemente (siehe \autoref{lst:content_imple_hash_print}) ist eigentlich selbsterklärend.\\
\lstinputlisting[style=custom, language=php, caption={/modules/form/HashGenerator.php; \texttt{check()}; Zeilen 47-72}, label={lst:content_imple_hash_print}, firstline=47, lastline=72, firstnumber=47]{sources/token/HashGenerator.php}
Die \texttt{check()}-Methode liest aus den abgesendeten Daten den Namen der Session aus und konstruiert mit ihm eine virtuelle Session. Sollte der Name nicht im Formular angegeben sein, wird der Name aus dem Konstruktor der \texttt{HashGenerator}-Klasse übernommen.\\
Existiert keine virtuelle Session mit dem Namen, kann davon ausgegangen werden, dass das Formular fehlerhaft, beziehungsweise nicht mehr aktuell ist. Es wird also eine Exception geworfen.\\
Nun werden die IDs verglichen, stimmt die ID des Konstruktors nicht mit der ID der virtuellen Session überein, so passt die Session nicht auf dieses Formular. Es wird eine Exception geworfen.\\
Als Nächstes werden die Hashes überprüft. Stimmt der mitgesendete Hash mit dem in der Session überein, so können wir davon ausgehen, dass derselbe passt. Ansonsten wird eine Exception geworfen.\\
Ist die Session älter als 30 Minuten, so ist das Formular nicht mehr gültig. Es wird eine Exception geworfen.\\
Die Session wird gelöscht, da sie nun nicht mehr benötigt wird.

\subsubsection{Anwendung}
Um diese Klasse anzuwenden, muss man zwischen zwei Teilen unterscheiden:
\begin{itemize}
	\item Das Erstellen des Formulars
	\item und das Auswerten desselben.
\end{itemize}
Auch wenn man diese beiden Vorgänge getrennt betrachten muss, sollte man sie nicht in verschiedenen Dateien verwenden.\\
\begin{lstlisting}[style=custom, language=php, caption={Anwendung von \texttt{HashGenerator}},label={lst:content_imple_hash_example}]
<?php
	// Laden des Basis-Systems
	include_once("../config.php");
	include_once(ROOT_LOCATION . "/modules/general/Main.php");
	
	// Laden des HashGenerators
	include_once(ROOT_LOCATION . "/modules/form/HashGenerator.php");
	
	// Erstellen einer Instanz von HashGenerator
	$hashGenerator = new HashGenerator("FormularName", __FILE__);
	
	// Unterscheidung zwischen:
	//   - Erstellung
	//   - Auswertung
	if (isset($_POST['sent'])) {
		// Auswertung
		
		try {
			$hashGenerator->check();
			
			// Wenn die Ausführung hier ankommt, ist kein Fehler aufgetreten,
			// wir können mit der Verarbeitung fortfahren.
		} catch (Exception $e) {
			// An Error occurred.
			// We can handle it here.
		}
	} else {
		// Erstellung
		
		// Generiere des Hashes
		$hashGenerator->generate();
		
		// Design "default" und Seiten-Name "Test-Seite"
		pageHeader("default", "Test-Seite");
?>
<!-- Definition des Formulars -->
<form action="?" method="POST">
	<?php
		// Ausgabe der versteckten Felder
		$hashGenerator->printForm();
	?>
	<input type="sumbit" name="sent" type="1" />
</form>
<?php
		pageFooter();
	}
?>
\end{lstlisting}
Zu beachten ist, dass bei Verwendung der \texttt{HashGenerator}-Klasse das \texttt{method}-Attribut des Formulars auf \texttt{POST} gesetzt sein muss.

\subsubsection{Hash-Check}
Für eine noch bequemere Abfrage des Hashes wird in der Datei /modules/form/hashCheckFail.php eine Funktion \texttt{HashCheck(\$hashgenerator)} definiert. Diese erwartet als Parameter eine Instanz der Klasse \texttt{HashGenerator}. Im Falle, dass der Hash passt, returnt die Funktion, wenn nicht, schreibt sie einen HTTP-Header, der den Webbrowser dazu bringt, die Seite mit dem GET-Parameter \texttt{?error} neuzuladen. Danach wird die Ausführung beendet.\\
Zusätzlich ist eine Funktion \texttt{HashFail()} definiert, die eben diesen Parameter überprüft und eine Fehlermeldung ausgibt, sofern der Parameter vorhanden ist.
\subsection{Stundenplan-Generierung}
\subsubsection{Standard-Stundenplan}
Aus dem mittels Datenbankabfrage erzeugten Objekten wird ein zweidimensionales Array mit der Form \texttt{\$hours[Stunde][Tag]} erstellt. (\autoref{lst:content_imple_timetables_array}).

\lstinputlisting[style=custom, language=php, caption={timetables/index.php; Umstrukturierung Array}, label={lst:content_imple_timetables_array},firstnumber=98,firstline=98,lastline=125]{sources/timetables/index.php}
\textbf{zu \autoref{lst:content_imple_timetables_array} :}
Zuerst wird kontrolliert ob zu dieser Stunde bereits ein Eintrag vorhanden ist(Zeile 100). Wenn dies nicht der Fall ist wird an dieser Stelle ein neues Array in dem anderen Array erzeugt.\\
Anschließend wird kontrolliert ob zu dieser Stunde, an diesem Tag bereits ein Eintrag vorhanden ist. Ist dies der Fall, dann wird ebenso kontrolliert ob der bereits eingetragene und der neue Eintrag für den selben Unterrichtsgegenstand gelten. Trifft dies nicht zu wird das neue Fach durch ein \enquote{|} getrennt an das bereits eingetragene Fach angehängt und das Popup erweitert. (Zeile 105-109). Ansonsten wird nur das Popup erweitert(Zeile 112-114).\\
Ist zu dieser Stunde und diesem Tag kein Eintrag vorhanden,wird einer erstellt und das dementsprechende Popup erzeugt. (Zeile 118 - 123). Dabei wird zwischen Lehrern und Schülern unterschieden.
\\\\
Anschließend wird für jeden Tag der folgende Ablauf durchgeführt:\\\\
Für den Fall, dass der modifizierte Stundenplan ausgewählt ist, werden zudem für den jeweiligen Tag die fehlenden Klassen abgerufen. Diese werden in ein Array geladen mit der Form \$missingClasses[Stunde][Klassenname] (\autoref{lst:content_imple_timetables_missing_classes_get}).\\

\lstinputlisting[style=custom, language=php, caption={timetables/index.php; Fehlende Klassen abrufen}, label={lst:content_imple_timetables_missing_classes_get},firstnumber=360,firstline=360,lastline=383]{sources/timetables/index.php}

 Anschließend wird dieses Array durchlaufen und kontrolliert, ob eine im \$hours-Array eingetragenene Klasse fehlt. Ist dies der Fall, wird dieses Feld geleert und als Popup der Grund eingetragen, aus welchem die Klasse fehlt. (\autoref{lst:content_imple_timetables_missing_classes_insert})
 
\lstinputlisting[style=custom, language=php, caption={timetables/index.php; Fehlende Klassen eintragen}, label={lst:content_imple_timetables_missing_classes_insert},firstnumber=136,firstline=136,lastline=150]{sources/timetables/index.php}
Ebenso werden natürlich die Supplierungen des aktuellen Tages abgerufen und durchlaufen. \\
Zuerst wird überprüft, ob die Supplierung eine hinzugefügt Stunde ist. Ist dies der Fall, wird sie wie eine normale Stunde eingetragen, jedoch mit der Klasse \enquote{changed}, wodurch der Hintergrund weiß und die Schrift schwarz wird.\\
Wenn dies nicht der Fall ist, wird kontrolliert, ob eine Stundenlöschung vorliegt. Sollte dieser Fall vorliegen, wird zudem kontrolliert, ob dies die einzige Stunde mit diesem Fach zu dieser Zeit ist. Hiermit soll verhindert werden, dass, wenn bei einer Klassenteilung der Unterricht von nur einer Gruppe ausfällt, im Stundenplan der gesamte Eintrag nicht angezeigt wird. Wenn es die einzige Stunde ist, wird der Eintrag aus dem Stundenplan gelöscht, die Klasse des Feldes auf \enquote{changed} geändert und im Popup das Kommentar der Supplierung angezeigt.\\
Sollte es keine Stundenlöschung sein, wird kontrolliert, ob eine Verschiebung vorliegt. Hierbei muss aus der ursprünglichen Stunde dieser Eintrag entfernt und in der neuen Stunde eingefügt werden. Bei beiden Stunden wird die Klasse des Feldes auf \enquote{changed} gestellt.\\
Sollte keiner der vorherigen Fälle zutreffen, muss es sich um eine \enquote{echte} Supplierung handeln. Dies bedeutet, dass bei der Eingabe nicht die Option \enquote{freie Eingabe} gewählt wurde. In diesem Fall wird im Popup bei der alten Stunde statt der Klasse bzw. Lehrkraft und dem Raum ein \enquote{-} dargestellt und das neue Fach, mit den neuen Einträgen, eine Zeile darunter geschrieben. In der Tabelle selber wird jedoch nur das neue Fach angezeigt.



\subsubsection{Alle Stundenpläne}
Zuerst wird überprüft, welche Berechtigungsstufe der User besitzt. Ein Schüler hat kein Recht diese Funktion zu nutzen und ein Lehrer darf nur die Klassenstundenpläne abrufen.\\

\lstinputlisting[style=custom, language=php, caption={timetables/all/index.php; Berechtigung; Auswahl}, label={lst:content_imple_timetables_missing_classes_insert},firstnumber=13,firstline=13,lastline=26]{sources/timetables/all/index.php}

Anschließend werden anhand der Eingabe die gewünschten Stunden abgerufen und, gleich wie beim Standard-Stundenplan, daraus der gewünschte Stundenplan generiert.
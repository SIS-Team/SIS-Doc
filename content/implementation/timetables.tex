\subsection{Stundenplan-Generierung}

Aus dem mittels Datenbankabfrage erzeugten Objekten wird ein zweidimensionales Array mit der Form \$hours[Stunde][Tag] erstellt. (\autoref{lst:content_imple_timetables_array}).

\lstinputlisting[style=custom, language=php, caption={timetables/index.php}, label={lst:content_imple_timetables_array},firstnumber=98,firstline=98,lastline=125]{sources/timetables/index.php}

Anschließend wird für jeden Tag der folgende Ablauf durchgeführt:\\\\

Für den Fall das der modifizierte Stundenplan ausgewählt ist werden zudem für den jeweiligen Tag die fehlenden Klassen abgerufen. Diese werden in ein Array geladen mit der Form \$missingClasses[Stunde][Klassenname]. Anschließend wird dieses Array durchlaufen und kontrolliert ob eine im \$hours-Array eingetragenene Klasse fehlt. Ist dies der Fall, wird dieses Feld geleert und als Popup der Grund eingetragen, aus dem die Klasse fehlt. \\\\
Ebenso werden natürlich die Supplierungen des aktuellen Tages abgerufen. \\
Zuerst wird überprüft ob die Supplierung eine hinzugefügt Stunde ist. Ist dies der Fall wird sie wie eine normale Stunde eingetragen jedoch mit der Klasse \enquote{changed} wodurch der Hintergrund weiß und die Schrift schwarz wird.\\
Wenn dies nicht der Fall ist, wird kontrolliert ob eine Stundenlöschung vorliegt. Sollte dieser Fall vorliegen, wird zudem kontrolliert, ob dies die einzige Stunde mit diesem Fach zu dieser Zeit ist. Hiermit soll verhindert werden, dass, wenn bei einer Klassenteilung der Unterricht einer Gruppe ausfällt, im Stundenplan der gesamte Eintrag nicht angezeigt wird. Wenn es die einzige Stunde ist wird der Eintrag aus dem Stundenplan gelöscht, die Klasse des Feldes auf \enquote{changed} geändert und im Popup das Kommentar der Supplierung angezeigt.\\
Sollte es keine Stundenlöschung sein, wird kontrolliert ob eine Verschiebung vorliegt. Hierbei muss aus der ursprünglichen Stunde dieser Eintrag entdfernt werden und in der neuen Stunde eingefügt werden. Bei beiden Stunden wird die Klasse des Feldes auf \enquote{changed} gestellt.\\
Sollte keiner der vorherigen Fälle zutreffen, muss es sich um eine \enquote{richtige} Supplierung handeln. Dies bedeutet, dass bei der Eingabe nicht die Option \enquote{freie Eingabe} gewählt wurde. In diesem Fall wird im Popup bei der alten Stunde statt der Klasse\ Lehrkraft und dem Raum ein \enquote{-} dargestellt und das neue Fach mit den neuen Einträgen eine Zeile darunter geschrieben. In der Tabelle selber wird jedoch nur das neue Fach angezeigt.

% Wie werden die Stundenpläne (normal und modifiziert) generiert?
% 
% Hier dürfen auch auch Sourcecode-Teile vorkommen.
% Wenn Sourcecodes: jeweilge File in den Ordner /sources/ in einen Unterordner packen und mit folgendem Befehl includieren:
%
%
% 
%
% Als weitere Eigenschaft kannst du die Zeilen angeben: [firstline=300, lastline=500]
% Damit nicht alles reinkopiert wird.
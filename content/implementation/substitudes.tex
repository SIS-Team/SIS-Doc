\subsection{Supplierplan-Generierung}

Zunächst wird überprüft ob der aktuelle Supplierplan oder der \enquote{folgende} Supplierplan ausgewählt wurde. Je nach gewählter Option beginnt die Zählvariable bei 0 oder 2.\\
Die Ansicht ist in zwei Bereiche (divs) unterteilt. Jedes div stellt einen Tag dar.\\
Für jeden Tag wird der folgende Ablauf durchgeführt:\\\\
Zuerst wird kontrolliert ob das Datum, welches mittels der Zählvariable und dem aktuellen Datum erzeugt wurde, ein Wochentag ist. Sollte es sich um ein Wochenende handeln, wird der darauf folgende Montag verwendet.\\
Anschließend wird zwischen Lehrern und Schülern bzw. Administratoren unterschieden. Dieser Teilung liegt zu Grunde, dass ein Lehrer andere Informationen sieht, als ein Schüler.
\\
Bei einem Schüler werden jene Supplierungen ausgegeben, welche die Klasse betreffen, der er zugeordent ist. Ein Lehrer sieht die Supplierungen, die ihn betreffen, ein Abteilungsleiter (Administrator) die Supplierungen seiner Abteilung und ein Superuser(root) sieht alle Supplierungen.\\
Die aus der Datenbank erhaltenen Datensätze werden anschließend nach Klasse (nur bei Administrator oder root) und Startstunde sortiert in einer Tabelle ausgegeben.
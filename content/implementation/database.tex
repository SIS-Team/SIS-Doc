\subsection{Datenbank}
% Implementierung der Datenbank
Die Implementierung der Datenbank wurde mit dem gratis erhältlichen Tool \textit{phpMyAdmin} vorgenommen. Unsere Wahl viel auf dieses Tool, da es einfach zu verwenden ist und nicht installiert werden muss. Die andere Möglichkeit wäre gewesen, dass wir die Datenbank mittels \textit{PHP} erstelle, dies hätte jedoch wesentlich mehr Zeit benötigt. Eine andere Möglichkeit wäre gewesen, dass wir mittels SSH und dem Kommandozeilen-Tool für MySQL die Datenbank erstellt hätten, dies war uns jedoch nicht möglich, da wir keinen SSH-Zugriff bekamen.
\subsubsection{Aufbau}
% Genaue Auflistung der Tabellen, Spalten, Kommentare, etc.
%
% @marco: Du hast das ja gemacht. Willst du das schreiben?
% Ich würde sagen, du schreibst da einfach, bis dir nichts mehr einfällt, und dann mentionst du mich in einem commit und ich schreib weiter.
Wie schon im Abschnitt \autoref{sec:content_solution_db} aufgelistet wurden die Tabellen dementsprechend erstellt. In diesem Abschnitt werden die Tabellen mit ihren Spalten und Datentypen aufgelistet.\\
\\
\paragraph{Klassen\\}
In dieser Tabelle werden alle Klassen gespeichert. Der Tabellenname lautet \textit{classes}. Diese Tabelle umfasst insgesamt 6 Spalten.
\begin{table}[H]
\centering
\begin{tabular}{p{2.5 cm}p{2.5 cm}p{10 cm}}
   \toprule
   \textbf{Spalte} & \textbf{Datentyp} & \textbf{Beschreibung} \\
   \midrule
          ID & Integer & Auto increment - Primärschlüssel  \\
          \hline
          name & Text & Name der Klasse \\
          \hline
          sectionFK & Integer & Fremdschlüssel auf die Abteilungs-Tabelle \\
          \hline
          teacherFK & Integer & Fremdschlüssel auf die Lehrer-Tabelle\\
          \hline
          roomFK & Integer & Fremdschlüssel auf die Raum-Tabelle\\
          \hline
          invisible & Boolean & Ob Eintrag sichtbar oder nicht\\
   \bottomrule
\end{tabular}
\caption{Klassen-Tabelle}
\end{table}

\paragraph{Display-Modus\\}
In dieser Tabelle sind die Modi für die Ein- und Ausschaltzeiten für die Monitore gespeichert. Der Tabellenname lautet \textit{displayMode}. Diese Tabelle umfasst insgesamt 2 Spalten.
\begin{table}[H]
\centering
\begin{tabular}{p{2.5 cm}p{2.5 cm}p{10 cm}}
   \toprule
   \textbf{Spalte} & \textbf{Datentyp} & \textbf{Beschreibung} \\
   \midrule
          ID & Integer & Auto increment - Primärschlüssel  \\
          \hline
          name & Text & Name des Modus \\
   \bottomrule
\end{tabular}
\caption{Display-Modus-Tabelle}
\end{table}

\paragraph{Stunden\\}
In dieser Tabelle werden alle Stunden gespeichert, mit deren Anfangs- und Endzeiten. Um nicht zusätzlich die Wochentage speichern zu müssen, werden dise zu den Stunden gespeichert. Deshalb ist eine gewisse Redundanz der Zeiten gegeben, diese nahmen wir jedoch in kauf. Zu einem Problem wird dies dann, wenn sich zum Beispiel eine Uhrzeit ändern sollte. Der Tabellenname lautet \textit{hours}. Diese Tabelle umfasst insgesamt 6 Spalten.
\begin{table}[H]
\centering
\begin{tabular}{p{2.5 cm}p{2.5 cm}p{10 cm}}
   \toprule
   \textbf{Spalte} & \textbf{Datentyp} & \textbf{Beschreibung} \\
   \midrule
          ID & Integer & Auto increment - Primärschlüssel  \\
          \hline
          weekday & Text & Wochentag - z.B.: Montag  \\
          \hline
          weekdayShort & Text & Wochentag-Kürzel z.B.: Mo   \\
          \hline
          hour & Integer & Stunde z.B: 1 oder 12  \\
          \hline
          startHour & Time & Anfangsstunde z.B.: 8:00  \\
          \hline
          endHour & Time & Endstunde z.B.: 8:50 \\
   \bottomrule
\end{tabular}
\caption{Stunden-Tabelle}
\end{table}

\paragraph{Stundenplan\\}
Für den Stundenplan sind 2 Tabellen notwendig. In einer Tabelle werden die Basisstunden einer Klasse gespeichert. In der anderen Tabelle werden die Baisstunden mit den anderen Informationen verknüpft. (siehe \autoref{sec:content_solution_db})\\

\subparagraph{Basis-Stundenplan\\}
In dieser Tabelle werden die Basisstunden gespeichert. Der Tabellenname lautet \textit{lessonsBase}. Diese Tabelle umfasst insgesamt 4 Spalten.

\begin{table}[H]
\centering
\begin{tabular}{p{2.5 cm}p{2.5 cm}p{10 cm}}
   \toprule
   \textbf{Spalte} & \textbf{Datentyp} & \textbf{Beschreibung} \\
   \midrule
          ID & Integer & Auto increment - Primärschlüssel  \\
          \hline
          classFK & Integer & Fremdschlüssel auf die Klassen-Tabelle   \\
          \hline
          startHourFK & Integer & Fremdschlüssel auf die Stunden-Tabelle  \\
          \hline
          endHourFK & Integer & Fremdschlüssel auf die Stunden-Tabelle  \\
   \bottomrule
\end{tabular}
\caption{Basis-Stundenplan-Tabelle}
\end{table}

\subparagraph{Eigentlicher Stundenplan\\}
In dieser Tabelle werden die Basisstunden mit zusätzlichen Informationen zusammengeführt und gespeichert. Der Tabellenname lautet \textit{lessons}. Diese Tabelle umfasst insgesamt 6 Spalten.

\begin{table}[H]
\centering
\begin{tabular}{p{2.5 cm}p{2.5 cm}p{10 cm}}
   \toprule
   \textbf{Spalte} & \textbf{Datentyp} & \textbf{Beschreibung} \\
   \midrule
          ID & Integer & Auto increment - Primärschlüssel  \\
          \hline
          lessonBaseFK & Integer & Fremdschlüssel auf die Basis-Stundenplan-Tabelle   \\
          \hline
	      roomFK & Integer & Fremdschlüssel auf die Raum-Tabelle   \\
	      \hline
          teachersFK & Integer & Fremdschlüssel auf die Lehrer-Tabelle   \\
          \hline
          subjectFK & Integer & Fremdschlüssel auf die Fächer-Tabelle  \\
          \hline
          comment & Text & Kommentar zur Stunde  \\
   \bottomrule
\end{tabular}
\caption{Stundenplan-Tabelle}
\end{table}

\paragraph{Logs\\}
Für das Speichern der Logs wurden insgesamt 5 Tabellen benötigt. Diese teilweise miteinander verknüpft sind.

\subparagraph{Logins\\}
In dieser Tabelle werden jegliche Anmeldeversuche an unsere Seite gespeichert. Außerdem wird hier auch gespeichert, ob die Anmeldung fehlschlug oder funktionierte. Der Tabellenname lautet \textit{logsLogins}. Diese Tabelle umfasst insgesamt 6 Spalten.

\begin{table}[H]
\centering
\begin{tabular}{p{2.5 cm}p{2.5 cm}p{10 cm}}
   \toprule
   \textbf{Spalte} & \textbf{Datentyp} & \textbf{Beschreibung} \\
   \midrule
          ID & Integer & Auto increment - Primärschlüssel  \\
          \hline
          time & Integer & Zeit als Unix-Timestamp   \\
          \hline
	      user & Text & Loginname   \\
	      \hline
          userAgent & Text & User Agent des Browsers   \\
          \hline
          ip & Text & IP-Adresse  \\
          \hline
          success & Integer & 1-Erfolgreich; 0-Nicht Erfolgreich  \\
   \bottomrule
\end{tabular}
\caption{Logins-Tabelle}
\end{table}

\subparagraph{Basis-Log\\}
In dieser Tabelle werden die Log-Informationen miteinander verknüpft. Hier werden alle Bewegungen in der Seite einer Session zugewiesen. Der Tabellenname lautet \textit{logsMain}. Diese Tabelle umfasst insgesamt 5 Spalten.

\begin{table}[H]
\centering
\begin{tabular}{p{2.5 cm}p{2.5 cm}p{10 cm}}
   \toprule
   \textbf{Spalte} & \textbf{Datentyp} & \textbf{Beschreibung} \\
   \midrule
          ID & Integer & Auto increment - Primärschlüssel  \\
          \hline
          time & Integer & Zeit als Unix-Timestamp   \\
          \hline
	      connFK & Integer & Fremdschlüssel auf die \textit{logsUSConn}-Tabelle   \\
	      \hline
          site & Text & Aufgerufene Seite   \\
          \hline
          params & Text & Angehängte Parameter an die Seite  \\
   \bottomrule
\end{tabular}
\caption{Basis-Log-Tabelle}
\end{table}

\subparagraph{User-Session-Log\\}
Da ein User mehrere Sessions und eine Session hintereinander mehrere User haben kann, musste eine Verbindungstabelle dieser beiden Informationen erstellt werden. Diese Daten werden in dieser Tabelle gespeichert. Der Tabellenname lautet \textit{logsUSConn}. Die Tabelle umfasst insgesamt 5 Spalten.

\begin{table}[H]
\centering
\begin{tabular}{p{2.5 cm}p{2.5 cm}p{10 cm}}
   \toprule
   \textbf{Spalte} & \textbf{Datentyp} & \textbf{Beschreibung} \\
   \midrule
          ID & Integer & Auto increment - Primärschlüssel  \\
          \hline
          time & Integer & Zeit als Unix-Timestamp   \\
          \hline
	      sessionFK & Integer & Fremdschlüssel auf die Session-Logs-Tabelle   \\
	      \hline
          userFK & Integer & Fremdschlüssel auf die User-Logs-Tabelle   \\
          \hline
          ip & Text & IP-Adresse  \\
   \bottomrule
\end{tabular}
\caption{User-Session-Tabelle}
\end{table}

\subparagraph{Session-Log\\}
Alle eröffneten Sessions mit dem Server werden hier gespeichert. Diese werden später in der Tabelle \textit{logsUSConn} mit dem User zusammengeführt. Der Tabellenname lautet \textit{logsSessions}. Die Tabelle umfasst insgesamt 5 Spalten.

\begin{table}[H]
\centering
\begin{tabular}{p{2.5 cm}p{2.5 cm}p{10 cm}}
   \toprule
   \textbf{Spalte} & \textbf{Datentyp} & \textbf{Beschreibung} \\
   \midrule
          ID & Integer & Auto increment - Primärschlüssel  \\
          \hline
          time & Integer & Zeit als Unix-Timestamp   \\
          \hline
	      PhpSessIDOrig & Text & Erste Session ID   \\
	      \hline
          userAgent & Text & User Agent des Browsers   \\
          \hline
          PhpSessIDAct & Text & Aktuelle Session ID  \\
   \bottomrule
\end{tabular}
\caption{Session-Tabelle}
\end{table}

\subparagraph{User-Log\\}
In dieser Tabelle werden die User gespeichert, welche unsere Seite besuchen. Um bei den Statistiken, ohne LDAP-Abfrage, zu erkennen welcher Abteilung/Klasse ein Schüler angehört, speicern wir in diesem Zusammenhang die Klasse und die Abteilung des Schülers. Der Tabellenname lautet \textit{logsUsers}. Diese Tabelle umfasst insgesamt 5 Spalten. 

\begin{table}[H]
\centering
\begin{tabular}{p{2.5 cm}p{2.5 cm}p{10 cm}}
   \toprule
   \textbf{Spalte} & \textbf{Datentyp} & \textbf{Beschreibung} \\
   \midrule
          ID & Integer & Auto increment - Primärschlüssel  \\
          \hline
          LDPA & Text & LDAP-ID des Schülers/Lehrers   \\
          \hline
	      classesFK & Integer & Fremdschlüssel auf die Klassen-Tabelle   \\
	      \hline
          sectionsFK & Integer & Fremdschlüssel auf die Abteilungs-Tabelle   \\
          \hline
          isTeacher & Integer & 1-Ja, 0-Nein  \\
   \bottomrule
\end{tabular}
\caption{User-Session-Tabelle}
\end{table}

\paragraph{Fehlende Klassen\\}
In dieser Tabelle werden die als fehlend eingetragenen Klassen gespeichert. Der Tabellenname lautet \textit{missingClasses}. Die Tabelle umfasst insgesamt 7 Spalten.

\begin{table}[H]
\centering
\begin{tabular}{p{2.5 cm}p{2.5 cm}p{10 cm}}
   \toprule
   \textbf{Spalte} & \textbf{Datentyp} & \textbf{Beschreibung} \\
   \midrule
          ID & Integer & Auto increment - Primärschlüssel  \\
          \hline
          classFK & Integer & Fremdschlüssel auf die Klassen-Tabelle   \\
          \hline
	      startDay & Date & Datum des Starttages   \\
	      \hline
          startHourFK & Integer & Fremdschlüssel auf die Stunden-Tabelle   \\
          \hline
          endDay & Date & Datum des Endtages   \\
          \hline
          endHourFK & Integer & Fremdschlüssel auf die Stunden-Tabelle   \\
          \hline
          reason & Text & Grund des Fehlens  \\
   \bottomrule
\end{tabular}
\caption{Fehlende-Klasse-Tabelle}
\end{table}

\paragraph{Fehlende Lehrer\\}
In dieser Tabelle werden die als fehlend eingetragenen Lehrer gespeichert. Der Tabellenname lautet \textit{missingTeachers}. Die Tabelle umfasst insgesamt 7 Spalten.

\begin{table}[H]
\centering
\begin{tabular}{p{2.5 cm}p{2.5 cm}p{10 cm}}
   \toprule
   \textbf{Spalte} & \textbf{Datentyp} & \textbf{Beschreibung} \\
   \midrule
          ID & Integer & Auto increment - Primärschlüssel  \\
          \hline
          teacherFK & Integer & Fremdschlüssel auf die Lehrer-Tabelle   \\
          \hline
	      startDay & Date & Datum des Starttages   \\
	      \hline
          startHourFK & Integer & Fremdschlüssel auf die Stunden-Tabelle   \\
          \hline
          endDay & Date & Datum des Endtages   \\
          \hline
          endHourFK & Integer & Fremdschlüssel auf die Stunden-Tabelle   \\
          \hline
          reason & Text & Grund des Fehlens  \\
   \bottomrule
\end{tabular}
\caption{Fehlende-Lehrer-Tabelle}
\end{table}

\paragraph{Monitor Modus\\}
In dieser Tabelle werden die verschiedenen Betriebsarten des Monitors gespeichert. Der Tabellenname lautet \textit{monitorMode}. Diese Tabelle umfasst insgesamt 2 Spalten.

\begin{table}[H]
\centering
\begin{tabular}{p{2.5 cm}p{2.5 cm}p{10 cm}}
   \toprule
   \textbf{Spalte} & \textbf{Datentyp} & \textbf{Beschreibung} \\
   \midrule
          ID & Integer & Auto increment - Primärschlüssel  \\
          \hline
          name & Text & Name des Modus  \\
   \bottomrule
\end{tabular}
\caption{Monitor-Modus-Tabelle}
\end{table}

\paragraph{Fehlende Lehrer\\}
In dieser Tabelle sind alle Moonitore eingetragen, die in SIS registriet werden. Die Zuteilung der verschiedenen Modi wird in dieser Tabelle gespeichert. Der Tabellenname lautet \textit{monitors}. Die Tabelle umfasst insgesamt 12 Spalten.

\begin{table}[H]
\centering
\begin{tabular}{p{2.5 cm}p{2.5 cm}p{10 cm}}
   \toprule
   \textbf{Spalte} & \textbf{Datentyp} & \textbf{Beschreibung} \\
   \midrule
          ID & Integer & Auto increment - Primärschlüssel  \\
          \hline
          name & Text & Name des Monitors   \\
          \hline
	      text & Text & Text der am Monitor links unten angezeigt wird  \\
	      \hline
          modeFK & Integer & Fremdschlüssel auf die Monitor-Modus-Tabelle   \\
          \hline
          file & Text & Dateinamen der zugewiesenen Datei    \\
          \hline
          roomFK & Integer & Fremdschlüssel auf die Raum-Tabelle   \\
          \hline
          sectionFK & Integer & Fremdschlüssel auf die Abteilungs-Tabelle   \\
          \hline
	      time & Integer & Erstellzeit als Unix-Timestamp   \\
	      \hline
          displayModeFK & Integer & Fremdschlüssel auf die Display-Modus-Tabelle   \\
          \hline
          displayStartDaytime & Integer & Unix-Timestamp der Einschaltzeit   \\
          \hline
          siplayEndDaytime & Integer & Unix-Timestamp der Ausschaltzeit   \\
          \hline                    	      
          ip & Text & IP Adresse beim Erstellen  \\
   \bottomrule
\end{tabular}
\caption{Monitor-Tabelle}
\end{table}

\paragraph{Räume\\}
In dieser Tabelle werden die Räume gespeichert. Der Tabellenname lautet \textit{rooms}. Diese Tabelle umfasst 3 Spalten.

\begin{table}[H]
\centering
\begin{tabular}{p{2.5 cm}p{2.5 cm}p{10 cm}}
   \toprule
   \textbf{Spalte} & \textbf{Datentyp} & \textbf{Beschreibung} \\
   \midrule
          ID & Integer & Auto increment - Primärschlüssel  \\
          \hline
          name & Text & Name des Raumes  \\
          \hline        
          teacherFK & Integer & Fremdschlüssel auf die Lehrer-Tabelle \\
   \bottomrule
\end{tabular}
\caption{News-Tabelle}
\end{table}

\paragraph{Abteilungen\\}
In dieser Tabelle werden die Abteilungen gespeichert. Der Tabellenname lautet \textit{sections}. Die Tabelle umfasst 4 Spalten.

\begin{table}[H]
\centering
\begin{tabular}{p{2.5 cm}p{2.5 cm}p{10 cm}}
   \toprule
   \textbf{Spalte} & \textbf{Datentyp} & \textbf{Beschreibung} \\
   \midrule
          ID & Integer & Auto increment - Primärschlüssel  \\
          \hline
          name & Text & Name der Abteilung   \\
          \hline
	      short & Text & Abteilungskürzel   \\
	      \hline
          teacherFK & Integer & Fremdschlüssel auf die Lehrer-Tabelle   \\
   \bottomrule
\end{tabular}
\caption{Abteilungs-Tabelle}
\end{table}

\paragraph{Fächer\\}
In dieser Tabelle werden alle Fächer gespeichert. Der Tabellenname lautet \textit{subjects}. Die Tabelle umfasst 4 Spalten.

\begin{table}[H]
\centering
\begin{tabular}{p{2.5 cm}p{2.5 cm}p{10 cm}}
   \toprule
   \textbf{Spalte} & \textbf{Datentyp} & \textbf{Beschreibung} \\
   \midrule
          ID & Integer & Auto increment - Primärschlüssel  \\
          \hline
          name & Text & Name des Faches   \\
          \hline
	      short & Text & Fächer-Kurzbezeichnung   \\
	      \hline
          invisible & Boolean & Ob Eintrag sichtbar oder nicht   \\
   \bottomrule
\end{tabular}
\caption{Fächer-Tabelle}
\end{table}

\paragraph{Fehlende Lehrer\\}
In dieser Tabelle werden alle Supplierungen gespeichert. Der Tabellenname lautet \textit{substitudes}. Die Tabelle umfasst insgesamt 13 Spalten.

\begin{table}[H]
\centering
\begin{tabular}{p{2.5 cm}p{2.5 cm}p{10 cm}}
   \toprule
   \textbf{Spalte} & \textbf{Datentyp} & \textbf{Beschreibung} \\
   \midrule
          ID & Integer & Auto increment - Primärschlüssel  \\
          \hline
          time & Date & Datum der Supplierung   \\
          \hline
	      newSub & Integer & Ob die Stunde hinzugefügt wird  \\
	      \hline
          remove & Boolean & Ob die Stunde entfernt wird   \\
          \hline
          lessonFK & Integer & Fremdschlüssel auf die Stundenplan-Tabelle    \\
          \hline
          startHourFK & Integer & Fremdschlüssel auf die Stunden-Tabelle   \\
          \hline
          endHourFK & Integer & Fremdschlüssel auf die Stunden-Tabelle   \\
          \hline
	      teacherFK & Integer & Fremdschlüssel auf die Lehrer-Tabelle \\
	      \hline
          subjectFK & Integer & Fremdschlüssel auf die Fächer-Tabelle   \\
          \hline
          roomFK & Integer & Fremdschlüssel auf die Raum-Tabelle \\
          \hline
          classFK & Integer & Fremdschlüssel auf die Klassen-Tabelle \\
          \hline    
          comment & Text & Kommentar \\
          \hline                 	      
          move & Boolean & Ob die Stunde verschoben wird  \\
   \bottomrule
\end{tabular}
\caption{Supplierungs-Tabelle}
\end{table}

\paragraph{Lehrer\\}
In dieser Tabelle werden alle Lehrer gespeichert. Der Tabellenname lautet \textit{teachers}. Die Tabelle umfasst insgesamt 6 Spalten.

\begin{table}[H]
\centering
\begin{tabular}{p{2.5 cm}p{2.5 cm}p{10 cm}}
   \toprule
   \textbf{Spalte} & \textbf{Datentyp} & \textbf{Beschreibung} \\
   \midrule
          ID & Integer & Auto increment - Primärschlüssel  \\
          \hline
          name & Text & Name des Lehrers \\
          \hline
	      short & Text & Lehrerkürzel   \\
	      \hline
          display & Text & Kurzname des Lehrers   \\
          \hline
          sectionFK & Integer & Fremdschlüssel auf die Abteilungs-Tabelle   \\
          \hline
          invisible & Boolean & Ob Eintrag sichtbar oder nicht \\
   \bottomrule
\end{tabular}
\caption{Lehrer-Tabelle}
\end{table}

\subsubsection{Zugriff}
% Ich weiß nicht, ob es bekannt war, aber es gab eine Home-Made Lib für den Datenbankzugriff von Marco
% Die könnte man hier dokumentieren.
% 
Für die Datenbankanbindung siehe \gref{sec:content_imple_base_connect}.\\

%
% Und die High-Light-Anfragen, die nicht über Marco's Lib laufen.
%

%
% Hier dürfen auch auch Sourcecode-Teile vorkommen.
% Wenn Sourcecodes: jeweilge File in den Ordner /sources/ in einen Unterordner packen und mit folgendem Befehl includieren:
%
%
% \lstinputlisting[style=custom, language=php, caption={Dateiname}, label={lst:content_imple_timetables_labelname}]{sources/ordner/datei.php}
%
% Als weitere Eigenschaft kannst du die Zeilen angeben: [firstline=300, lastline=500]
% Damit nicht alles reinkopiert wird.
\subsection{Basis-System}
Das Basissystem ist relativ simple gehalten.\\
\subsubsection{Konfiguration}
Die erste - und wohl auch wichtigste - Datei des Basis-Systems ist die Datei /config.php, diese soll unter anderem das Problem des Wurzelverzeichnisses lösen (siehe \gref{sec:content_solutions_root}).
\lstinputlisting[style=custom, language=php, caption={/config.php}, label={lst:content_imple_base_config}]{sources/base/config.php}
In dieser werden drei Konstanten definiert:
\begin{description}
	\item[\texttt{BETA}] ist \texttt{true} solange das System im Testbetrieb ist.
	\item\texttt{[RELATIVE\_ROOT}] beinhaltet den relativen Pfad vom Grundverzeichnis des virtuellen Hosts (vorgegeben durch den Webserver) zum Grundverzeichnis des Projektes. Hierbei ist zubachten, dass das erste Querstrich-Zeichen wegzulassen ist.\\
	\textit{Beispiel:} Ist das Grundverzeichnis des virtuellen Hosts bereits das Wurzelverzeichnis des Projektes, so muss der String "'"' verwendet werden.\\
	\textit{Beispiel:} Befindet sich das Projekt im Ordner \enquote{foobar} unterhalb des Grundverzeichnisses des virtuellen Hosts, so ist der betreffende String "'\texttt{foobar/}"'.
	\item[\texttt{ROOT\_LOCATION}] beinhaltet den absoluten Pfad vom Grundverzeichnis des Servers zum Projektverzeichnis. Dieser wird aus dem \texttt{DOCUMENT\_ROOT}, also dem Grundverzeichnis des virtuellen Hosts, und dem relativen Pfad des Projekt-Ordners zum \texttt{DOCUMENT\_ROOT} gebildet.
\end{description}

\subsubsection{Main-File}
Die quasi Hauptdatei des Basis-Systems stellt die Datei /modules/general/Main.php dar. Diese bindet weitere wichtige Dateien des Grundsystems ein, dies soll das Einbinden der wichtigsten Module erleichtern.
\lstinputlisting[style=custom, language=java, caption={/modules/general/Main.php},  label={lst:content_imple_base_main}, firstline=8, lastline=12, firstnumber=8]{sources/base/Main.php}
Es werden folgende Module des Basis-Systems eingebunden:
\begin{itemize}
	\item CheckCookies, siehe \autoref{sec:content_imple_base_cookie}
	\item ForceHTTPS, siehe \autoref{sec:content_imple_base_https}
	\item SessionManager, siehe \autoref{sec:content_imple_base_session}
	\item Connect, siehe \autoref{sec:content_imple_base_connect}
	\item Site, siehe \autoref{sec:content_imple_base_site}
\end{itemize}

%
%
% \lstinputlisting[style=custom, language=php, caption={Dateiname}, label={lst:content_imple_timetables_labelname}]{sources/ordner/datei.php}
%
% Als weitere Eigenschaft kannst du die Zeilen angeben: [firstline=300, lastline=500]
% Damit nicht alles reinkopiert wird.
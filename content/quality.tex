Um die Qualität während der Entwicklung des Projekts einigermaßen zu erhalten, wurde der Code fortlaufend ausgetauscht. Dazu wurde GitHub verwendet. Durch den ständigen Austausch des Codes wird vermieden, dass Teile des Programmes doppelt entwickelt werden und dass bei Problemen andere Mitarbeiter sofort Einsicht in den Code haben.\\
Weiters wurde versucht bei der Programmierung auf einen einigermaßen einheitlichen Programmierstil zu achten, deshalb wurde auch grundsätzlich auf Englisch programmiert und die Dateien wurden alle englisch benannt. (Ausnahme: lokale Dateien der App)\\
Bei Fremd-Software wurde auf Seriosität geachtet. Zum Beispiel PhoneGap, das Framework zum Erstellen der App, wird von Adobe-Systems zur Verfügung gestellt. Dadurch ist bestätigt, dass es sich bei PhoneGap um ein seriöses Produkt handelt.\\
Um auch eine gewisse Sicherheit zu gewährleisten, werden alle Verbindungen über HTTPS, also verschlüsselt, aufgebaut. Dadurch werden die Daten sicher an den Server übertragen.\\
Auch aus Gründen der Sicherheit wurde auf die MySQL-Verwaltungssoftware PhpMyAdmin verzichtet. Diese könnte eine potenzielle Sicherheitslücke darstellen.\\
\\
Ein weiteres Qualitätsmerkmal stellt die Bedienungsfreundlichkeit dar. Da dieses Projekt davon lebt, dass es von möglichst vielen Personen genutzt wird, ist es wichtig, dass der Nutzer die Software gerne nutzt.\\
Darum muss darauf geachtet werden, dass die Bedienung für den Benutzer möglichst einfach und unkompliziert ist.
Aber auch die Bedienung für den Administrator muss trotz der vielen Einstellungen und Menüpunkte übersichtlich bleiben.\\
 

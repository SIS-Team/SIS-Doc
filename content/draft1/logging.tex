\subsection{Logging}
% TODO flo
% Was wird geloggt? Warum? Ab wann? Wie?

Als Seitenbetreiber ist man gut beraten, zu speichern, was wann passiert. Dies hat unter anderem den Grund, dass im Falle eines gröberen Sicherheitsproblems zurückverfolgt werden kann, wer das Problem verursacht hat. Der verwendete Webserver Apache hat bereits von sich aus Funktionen zur Dokumentation der Seitenaufrufe. Diese haben allerdings den Nachteil, dass sie nur die Aufrufe mit Zeit und IP und einigen weiteren Daten mitschreiben. Es kann also nur mit längeren Umwegen festgestellt werden, welcher Benutzer zu dieser Zeit angemeldet war. Zusätzlich ist der Zugang zu den Log-Daten nicht sehr komfortabel, da diese nur in Textdateien gespeichert werden. Deshalb wurde ein zusätzliches System geschaffen, um die Zugriffe effizient zu dokumentieren.\\
Als Speichermedium wird MySQL gewählt. Dies hat den Vorteil, dass die Datensätze effizient durchsucht werden können (dies ist vorteilhaft für die Generierung von etwaigen Statistiken). Ein Nachteil ist, dass die Log-Daten bei einem SQL-Injection-Angriff gelöscht oder modifiziert werden können (aus diesem Grund wird zusätzlich noch der Standard-Apache-Log eingesetzt).\\
Es soll erst geloggt werden, nachdem der Benutzer angemeldet ist. Andere Benutzer, die eventuell nur zufällig auf die Seite gelangen, könnten sonst die Benutzungs-Statistiken verfälschen. Für unangemeldete Benutzer reichen im Problemfall sowieso die Daten des Apache-Logs.\\
\\
Damit auch größere Zusammenhänge, wie zum Beispiel, dass mehrere Benutzer sich nacheinander auf dem gleichen PC angemeldet haben, erfasst werden können, muss eine flexible Datenbankstruktur geschaffen werden, die solche Dinge berücksichtigt.\\
\\
Ein guter Anfang ist das Anlegen einer Tabelle für alle PHP-Sessions. Dies ist allerdings alleine schon ein Problem, denn, wie später unter \gref{sec:content_imple_base_session} beschrieben wird, ändert sich die Session-ID nach jedem Neuladen der Seite. Die Lösung ist recht einfach: Es wird die ursprüngliche Session-ID in der Session gespeichert. In der Datenbank wird diese dann verglichen.\\
Das Problem setzt sich fort, wenn man bedenkt, das es durchaus möglich ist, dass zwei Sessions in einem zeitlichen Abstand die gleich ID bekommen. Dies kann einfach umgangen werden, indem der Timestamp der Eröffnung der Session mitgespeichert und dann später ebenfalls in die Datenbank geschrieben und verglichen wird.\\
Auch der User-Agent-String wird in dieser Tabelle gespeichert, da eine PHP-Session üblicherweise fix mit einem Browser verknüpft ist.\\
\\
Es ist nicht unwahrscheinlich, dass sich in einer Session mehrere verschiedene Benutzer anmelden. Um diesen Fall abzudecken, wird zunächst eine Tabelle für die Benutzer erstellt. Immer wenn sich ein User zum ersten Mal anmeldet, wird ein neuer Eintrag in dieser Tabelle erstellt. Es bietet sich an, die Benutzer-Tabelle mit den bereits vorhandenen Tabellen für die Abteilung und die Klassen zu verknüpfen. Zusätzlich wird noch eine Spalte vorgesehen, ob es sich um einen Lehrer handelt, oder nicht.\\
Nun wird die Session- mit der Benutzer-Tabelle verknüpft. Bei jeder Anmeldung wird hier ein neuer Eintrag mit dem Zeitpunkt des Logins angelegt. Somit kann nun jederzeit festgestellt werden, welcher Benutzer wann mit welcher Session eingeloggt war. Hier wird auch gespeichert, welche IP-Adresse der Benutzer zu diesem Zeitpunkt hatte.\\
\\
Nun kommt es zum eigentlichen Logging-Prozess: Es braucht eine weitere Tabelle, diese stellt die Haupttabelle für das Logging dar. Bei jedem Aufruf einer Seite, die den SessionManager einbindet, wird in dieser ein neuer Eintrag erstellt. Ein Eintrag besteht aus dem Fremdschlüssel der Verbindungstabelle zwischen User- und Session-Tabelle, der URL der aufgerufenen Seite, ihrer GET-Parameter und Fragmentbezeichnern und natürlich der Zeit des Aufrufs.\\
\\
Durch diesen eigentlich recht simplen Aufbau können bereits alle Bewegungen der einzelnen Benutzer auf der Seite (auch, wenn sich ein User in mehreren Webbrowsern zur selben Zeit angemeldet hat) nachvollzogen werden.\\
\\
Der größte Nachteil an diesen Überlegungen ist, dass zwar die Bewegungen der Benutzer, die bereits angemeldet sind, gespeichert werden können, aber die Login-Versuche in keiner Weise irgendwie gespeichert werden. Um dies zu lösen, wird eine weitere Tabelle eingeführt, die nur dazu da ist, zu speichern, welche IP-Adresse, um welche Zeit, mit welchem userAgent, mit welchem Benutzernamen sich versucht hat anzumelden, und ob der Anmeldevorgang erfolgreich war, oder nicht. Alle diese Daten müssen deshalb hier alle nocheinmal gespeichert werden, da in den anderen Tabellen möglicherweise noch gar keine Einträge für die jetzige Session pexistieren.\\
\\
\begin{figure}[H]
\centering
\resizebox{16cm}{!}{ 
	\fdot[scale=0.8]{images/dbLogs}{}
}
\caption{Log-Datenbank}
\end{figure}
Für die genaue Form der besprochenen Tabellen siehe \autoref{sec:content_imple_db_aufbau} Unterpunkt \enquote{Logs}.
 

\subsubsection{Statistiken}
% Das ist ein Unterpunkt von Logging
% Hier könnte man erwähnen, woraus Statistiken erstellt werden.
% Welche Lib wird verwendet? (Und warum?)
% Was werden für Statistiken erstellt?
% Wofür können sie eingesetzt werden?
%
% Und was dir sonst so einfällt.
% Hier kann man ganz viele Bilder rein tun. Die hast du wahrscheinlich eh schon alle im /images/screenshots/-Ordner

\subsubsection{Erkenntnisse}
% Hier kann man aktuelle Werte erwähnen und diskutieren (in Bezug auf Uhrzeiten, Wochenende, Geräte, etc).
%
% Vielleicht hier dazu schreiben, dass das zwar interessant ist, allerdings für das Projekt selbst eher unwichtig ist, wir wollten es aber trotzdem erwähnen. #oderso
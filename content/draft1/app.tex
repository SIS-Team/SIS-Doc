\subsection{App}
% Prinzip der App, Entwurf
\subsubsection{Aufbau}
Die SIS-App dient als Frontend für mobile Geräte. Bei der App handelt es sich um ein reines Ausgabemedium für alle Nutzer, es gibt keine Möglichkeit für Administratoren Supplierungen, Neuigkeiten oder ähnliches einzutragen.\\
Damit die Anmeldung für den Nutzer möglichst einfach ist, erfolgt diese mit den Novell-Zugangsdaten. Welche ähnlich wie bei der SIS-Webseite via LDAP überprüft werden.\\
In der App gibt es sämtliche Ausgaben die Standardbenutzer ohne spezielle Berechtigungen auch sehen. Es werden der eigene Stundenplan, inklusive unterrichtendem Lehrer und genutztem Raum, eine Tabelle mit allen Supplierungen und ein daraus resultierender Stundenplan für den jeweiligen Nutzer erzeugt.\\
Zusätzlich gibt es in der App auch eine Möglichkeit die Neuigkeiten die auf den Monitoren angezeigt werden anzusehen.\\

\subsubsection{Menüführung}

Da mobile Geräte wie zum Beispiel Smartphones oft kleine Displays haben, muss die Bedienung möglichst einfach gehalten und die App sehr übersichtlich gestaltet werden. Darum wurden für das Menü nur drei Menüpunkte ausgewählt (Stundenpläne, Supplierpläne und News) und diese mit großen Buttons umgesetzt.Weitere Menüpunkte, wie der „angepasste Stundenplan“ oder für Lehrer den Punkt „alle Stundenpläne“, wurden deshalb direkt auf der Stundenplanseite platziert, dadurch kommt man über den Menüpunkt Stundenpläne auf alle gewünschten Varianten des Stundenplans.\\
Damit die Tabellen für Stundenpläne und Supplierungen nicht zu unübersichtlich werden, werden nur die notwendigsten Informationen in die Tabelle geschrieben, um weitere Informationen zu erlangen, gibt es Popup-Fenster mit weiteren Daten, welche erscheinen wenn man auf die gewünschte Zelle in der Tabelle tippt.\\


\subsubsection{Konzept}
Zum erstellen einer App für Smartphones und Tablets wurde das Framework PhoneGap von Adobe genutzt. Dadurch konnten für die Entwicklung der App Webtechnologien, wie zum Beispiel HTML oder JavaScript, verwendet werden. Im Gegensatz zu herkömmlichen Web-Apps haben diese Applikationen aber den Nachteil, dass PHP-Seiten nicht direkt auf dem Gerät verwendet werden können, da die App lokal auf dem Gerät und nicht auf einem Webserver betrieben wird.
Deshalb muss man bei der Entwicklung einer App dieser Art die Applikationsdateien grundsätzlich in 2 Arten Unterteilen.\\
Die App-Dateien: Das sind jene Dateien welche direkt auf dem Gerät gespeichert, werden beziehungsweise als App gedownloadet werden (zum Beispiel HTML oder JavaScript).\\
Die API-Dateien: Das sind jene Dateien welche auf dem Sever gespeichert sind und welche der Applikation den Zugriff auf die benötigten Daten, wie zum Beispiel Stundenpläne, ermöglichen (dabei handelt es sich hauptsächlich um PHP-Dateien).\\
\\
App:	In der App werden alle Daten gespeichert welche nicht dynamisch geändert werden müssen, also Daten die für jeden Benutzer gleich sind, zum Beispiel die Rohtabelle auf der Stundenplanseite oder das gesamte Menü. Da sich diese Dinge nie ändern, wurden sie direkt in der App gespeichert, um Downloadvolumen und Ladezeit zu sparen.\\
Zu den Daten der App gehören aber auch Bilder für das App-Icon oder den Splashscreen (ein Bild welches beim Starten der App angezeigt wird).\\
Und eine wichtige Datei muss in der App noch enthalten sein. Die Datei config.xml, dabei handelt es sich um eine Konfigurationsdatei, in welcher alle wichtigen Informationen zur App stehen, wie zum Beispiel der Name der App, um welche Version es sich handelt, welche Berechtigungen die App besitzt, etc.\\
\\
API:	Eine API (Application Programming Interface) ist eine Programmierschnittstelle. In diesem Fall wird die API benötigt, um die Daten aus der Datenbank in die App zu laden. Als API bezeichnen wir in diesem Fall alle Dateien, welche auf dem Server gespeichert sind und von der App genutzt werden, zum Beispiel login.php oder timetables.php.
Dabei werden die Dateien in zwei Varianten unterschieden. Es gibt solche die reine PHP-Files sind (login.php und logout.php) und solche die zwar PHP-Code enthalten aber auch JavaScript-Code enthalten und später auch als JavaScript geladen werden.\\


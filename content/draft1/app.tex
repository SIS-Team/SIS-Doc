\subsection{App}
% Prinzip der App, Entwurf
\subsubsection{Aufbau}
Die SIS-App dient als Frontend für mobile Geräte. Bei der App handelt es sich um ein reines Ausgabemedium für alle Nutzer, es gibt keine Möglichkeit für Administratoren Supplierungen, Neuigkeiten oder ähnliches einzutragen.\\
Damit die Anmeldung für den Nutzer möglichst einfach ist, erfolgt diese mit den Novell-Zugangsdaten. Welche ähnlich wie bei der SIS-Webseite via LDAP überprüft werden.\\
In der App gibt es sämtliche Ausgaben die Standardbenutzer ohne spezielle Berechtigungen auch sehen. Es werden der eigene Stundenplan, inklusive unterrichtendem Lehrer und genutztem Raum, eine Tabelle mit allen Supplierungen und ein daraus resultierender Stundenplan für den jeweiligen Nutzer erzeugt.\\
Zusätzlich gibt es in der App auch eine Möglichkeit die Neuigkeiten die auf den Monitoren angezeigt werden anzusehen.\\

\subsubsection{Menüführung}

Da mobile Geräte wie zum Beispiel Smartphones oft kleine Displays haben, muss die Bedienung möglichst einfach gehalten und die App sehr übersichtlich gestaltet werden. Darum wurden für das Menü nur drei Menüpunkte ausgewählt (Stundenpläne, Supplierpläne und News) und diese mit großen Buttons umgesetzt.Weitere Menüpunkte, wie der \enquote{angepasste Stundenplan} oder für Lehrer den Punkt \enquote{alle Stundenpläne}, wurden deshalb direkt auf der Stundenplanseite platziert, dadurch kommt man über den Menüpunkt Stundenpläne auf alle gewünschten Varianten des Stundenplans.\\
Damit die Tabellen für Stundenpläne und Supplierungen nicht zu unübersichtlich werden, werden nur die notwendigsten Informationen in die Tabelle geschrieben, um weitere Informationen zu erlangen, gibt es Popup-Fenster mit weiteren Daten, welche erscheinen wenn man auf die gewünschte Zelle in der Tabelle tippt.\\


\subsubsection{Konzept}
Zum erstellen einer App für Smartphones und Tablets wurde das Framework PhoneGap von Adobe genutzt. Dadurch konnten für die Entwicklung der App Webtechnologien, wie zum Beispiel HTML oder JavaScript, verwendet werden. Im Gegensatz zu herkömmlichen Web-Apps haben diese Applikationen aber den Nachteil, dass PHP-Seiten nicht direkt auf dem Gerät verwendet werden können, da die App lokal auf dem Gerät und nicht auf einem Webserver betrieben wird.
Deshalb muss man bei der Entwicklung einer App dieser Art die Applikationsdateien grundsätzlich in 2 Arten Unterteilen.\\
Die App-Dateien: Das sind jene Dateien welche direkt auf dem Gerät gespeichert, werden beziehungsweise als App gedownloadet werden (zum Beispiel HTML oder JavaScript).\\
Die API-Dateien: Das sind jene Dateien welche auf dem Sever gespeichert sind und welche der Applikation den Zugriff auf die benötigten Daten, wie zum Beispiel Stundenpläne, ermöglichen (dabei handelt es sich hauptsächlich um PHP-Dateien).\\
\\
App:	In der App werden alle Daten gespeichert welche nicht dynamisch geändert werden müssen, also Daten die für jeden Benutzer gleich sind, zum Beispiel die Rohtabelle auf der Stundenplanseite oder das gesamte Menü. Da sich diese Dinge nie ändern, wurden sie direkt in der App gespeichert, um Downloadvolumen und Ladezeit zu sparen.\\
Zu den Daten der App gehören aber auch Bilder für das App-Icon oder den Splashscreen (ein Bild welches beim Starten der App angezeigt wird).\\
Und eine wichtige Datei muss in der App noch enthalten sein. Die Datei config.xml, dabei handelt es sich um eine Konfigurationsdatei, in welcher alle wichtigen Informationen zur App stehen, wie zum Beispiel der Name der App, um welche Version es sich handelt, welche Berechtigungen die App besitzt, etc.\\
\\
API:	Eine API (Application Programming Interface) ist eine Programmierschnittstelle. In diesem Fall wird die API benötigt, um die Daten aus der Datenbank in die App zu laden. Als API werden in diesem Fall alle Dateien bezeichnet, welche auf dem Server gespeichert sind und von der App genutzt werden, zum Beispiel login.php oder timetables.php.
Dabei werden die Dateien in zwei Varianten unterschieden. Es gibt solche die reine PHP-Files sind (login.php und logout.php) und solche die zwar PHP-Code enthalten aber auch JavaScript-Code enthalten und später auch als JavaScript geladen werden.\\

\paragraph{Stundenplan\\}
Die eigentliche Stundenplanseite ist nur eine einfache HTML-Seite, mit einer fast leeren Tabelle, in der nur die Wochentage und die Zeiten der Unterrichtsstunden stehen. Denn diese Informationen sind die einzigen Informationen die bei allen Nutzern gleich sind und dadurch, dass diese Informationen statisch in die Webseite eingebaut sind, sind sie lokal auf dem Gerät gespeichert und müssen nicht mehr heruntergeladen werden, dass spart Datenvolumen und Zeit.\\
Da es viele verschiedene Stundenpläne gibt, muss zuerst einmal überprüft werden um welchen Nutzer es sich handelt. Dabei wird als erstes überprüft ob sich ein Lehrer oder ein Schüler angemeldet hat. Hat sich ein Lehrer angemeldet, dann werden alle Stundenplan-Datensätze(Datensätze aus der Tabelle \enquote{lessons}), bei denen dieser Lehrer eingetragen ist abgerufen und in den Stundenplan eingetragen, handelt es sich jedoch um einen Schüler muss zuerst noch überprüft werden welche Klasse der Schüler besucht. Wenn die Klasse bekannt ist, werden, gleich wie bei den Lehrern, alle Stundenplan-Datensätzen, bei denen diese Klasse eingetragen ist, abgerufen und in den Stundenplan eingetragen.\\
Beim Abruf der Daten aus der Datenbank tritt wieder die selbe Problematik wie bei LDAP auf, der Abruf wird mit PHP gemacht, aber die Applikation arbeitet rein mit JavaScript. Deshalb wird genau wie beim Login wieder eine PHP-Datei als JavaScript geladen.\\
In dieser Datei werden zuerst mit PHP alle relevanten Stundenplandaten, also nur jene die den jeweiligen Schüler oder Lehrer betreffen, abgerufen und in ein PHP-Array gespeichert. Um die Daten möglichst einfach mit JavaScript zu verwenden wird dieses Array mit der Funktion jsonencode() in ein JSON-Objekt umgewandelt.\\
Dieses JSON-Objekt wird nun mit einem ECHO-Befehl direkt so in der Datei gespeichert, dass das Objekt nach dem Einlesen der JavaScript-Datei als Variable verfügbar ist.\\
Als nächstes müssen die Daten in die Stundenplantabelle eingetragen werden. Da die Daten in diesem Objekt nicht geordnet sind, müssen die Elemente dieses Objektes einzeln ausgelesen und analysiert werden. Dazu verwendet man am besten eine for-Schleife die das Objekt, Element für Element durchgeht. Bei jedem Element werden zuerst der Tag und die Unterrichtszeit überprüft, aus diesen beiden Informationen wird dann eine ID zusammengesetzt.\\
\begin{description}[style=nextline]
	\item[Zusammensetzung der ID:]
		Für die Wochentage werden Nummern von eins bis fünf verteilt(Mo = 1, Di = 2, etc.), eine zweite Zahl wird für 		die Unterrichtszeiten vergeben, die Nummer wird dadurch bestimmt, um die wievielte Unterrichtsstunde es sich 			an diesem Tag handelt. Diese beiden Zahlen werden zusammengesetzt und ergeben die ID der Zelle in die sie 				eingetragen werden.\\
		Beispiel:\\	
					Dienstag, 3.Stunde\\
					Dienstag $\hat{=}$ 2\\
					3.Stunde $\hat{=}$ 3\\
					ID = 23\\
\end{description}

Nach dem zusammensetzen der ID wird das Fach der entsprechenden Stunde in die Zelle mit der entsprechenden ID eingetragen. Die IDs werden den Zellen direkt beim Erstellen der Tabelle fix zugewiesen\\.
Da bei unserer Datenbankstruktur für eine Doppel-, oder Mehrfachstunde nur ein Eintrag besteht und nicht für jede Stunde ein einzelner Eintrag erstellt wird, muss für diesen Fall noch eine Extraroutine durchgeführt werden.\\
Deshalb werden die einzelnen Stunden mit Hilfe einer Schleife eingetragen, dabei wird bei jedem Durchlauf der Schleife überprüft ob die Endstunde(die letzte Stunde die noch zum jetzigen Element gehört)mit der aktuell eingetragenen Stunde übereinstimmt und wenn nicht wird die Schleife wiederholt und der Wert in der ID, der für die Stunde steht, wird inkrementiert. Bei einer Einzelstunde würden gleich beim ersten Durchlauf die aktuelle und Endstunde übereinstimmen, da die erste zugleich auch die letzte Unterrichtsstunde dieses Elements ist.\\
Bei einer Doppelstunde braucht es hingegen zwei Durchläufe bis eine Übereinstimmung vorliegt, dadurch wird das Fach auch in zwei Zellen eingetragen.\\
Ein weiteres Problem stellt die Abendschule dar. Da Tagesschüler keinen Unterricht nach 17:45 Uhr(letzte Tagesunterrichtsstunde: 16:55 – 17:45 Uhr(11.Stunde))haben wäre es nicht sinnvoll die ganze Tabelle mit den leeren Felder für die Abendschulstunden anzuzeigen. Deshalb wird bei Schülern überprüft ob eine Stunde nach 17:45 Uhr eingetragen ist, wenn ja dann wird davon ausgegangen, dass es sich um einen Abendschüler handelt und es wird nur der Teil ab der 12.Stunde angezeigt. Wird jedoch kein Eintrag nach 17:45 Uhr festgestellt, geht man davon aus, dass es sich um einen Tagesschüler handelt, und es werden nur die Tagesunterrichtsstunden angezeigt.\\
Bei Lehrern gibt es nicht nur diese beiden Fälle, sondern es gibt auch noch einen dritten Fall bei dem der Lehrer in der Tagesschule und in der Abendschule unterrichtet, in diesem Fall muss die vollständige Tabelle angezeigt werden.\\
Dazu wird bei der Überprüfung ob Stunden nach 17:45 Uhr eingetragen sind zusätzlich überprüft ob es sich um einen Schüler oder einen Lehrer handelt. Handelt es sich um einen Lehrer und es ist eine Unterrichtsstunde nach 17:45 Uhr eingetragen wird die ganze Tabelle angezeigt, da jeder Lehrer nach unseren Wissensstand auch Tagesstunden unterrichten sollte.\\

\paragraph{Supplierplan\\}
Für den Supplierplan existiert wieder eine eigene Website. Im Gegensatz zur Stundenplanseite kann die Tabelle hier nicht im Voraus erstellt werden, da nicht bekannt ist wie viele Supplierungen anfallen, und somit nicht bekannt ist wie viele Zeilen für die Tabelle benötigt werden. Das heißt, dass die Tabelle dynamisch, unter der Zuhilfenahme von JavaScript, erstellt werden muss. Nur der Tabellenkopf kann bereits in den HTML-Code integriert werden.\\
Auch beim Supplierplan müssen wieder Daten von der Datenbank abgerufen werden, das heißt auch hier wird wieder eine PHP-Datei als JavaScript geladen um an die Daten zu kommen. Dabei wird in der PHP-Datei wieder als erstes Überprüft ob es sich um einen Schüler oder einen Lehrer handelt.\\
Je nach dem werden dann die Supplierungsdatensätze(Datensätze aus der Tabelle \enquote{substitudes})des Lehrers oder der Klasse des Schülers abgerufen und in ein PHP-Array geschrieben. Dieses Array wird wie bereits beim Stundenplan mit der Funktion jsonencode() in ein JSON-Objekt umgewandelt und mit dem ECHO-Befehl als Variable in die Datei geschrieben, so dass die Variable nach dem einlesen in die Webseite, noch zur Verfügung steht.\\
Jetzt werden wieder mit Hilfe eine for-Schleife die einzelnen Elemente des JSON-Objektes betrachtet. Für jedes Element wird eine eigene Zeile in der Supplierungstabelle erstellt. Da die Tabelle jetzt dynamisch erstellt wird, müssen auch die IDs der Zellen, die zum eintragen in die Tabelle benötigt werden, erst eingetragen werden.\\
Diese ID wird aber anders zusammengesetzt als die ID beim Stundenplan. In diesem Fall wird eine Nummer für die Startstunde vergeben, eine zweite Nummer für den Fall das es sich um eine Mehrfachstunde handelt(damit jede einzelne Stunde eingetragen wird) und noch eine dritte Nummer um zu definieren um welche Information es sich handelt(Tag=1; Stunde=2; Fach=3; zu Supplieren durch = 4; Bemerkung=5).\\

\paragraph{Angepasster Stundenplan\\}
Beim angepassten Stundenplan müssen nun der Stundenplan und der Supplierplan kombiniert werden. So sollen zum Beispiel entfallene Stunden aus dem Stundenplan ausgetragen und supplierte Stunden markiert werden.\\
Dazu wird zuerst der Stundenplan, gleich wie der gewöhnliche Stundenplan, generiert. Danach werden die Supplierungen aus der Datenbank abgerufen, diese werden in eine Schleife alle einzeln überprüft. In der Schleife wird überprüft ob es sich um eine ausfallende, eine neu eingetragene, eine verschobene oder eine supplierte Stunde handelt.\\ Entsprechend des Ergebnisses dieser Überprüfung werden die Stunden ausgetragen, zusätzlich eingetragen oder verschoben.\\
Bei allen Zellen welche von einer Supplierung betroffen sind wird ein zusätzliches Attribut gesetzt. Dabei handelt es sich um das selbst gewählte Attribut \enquote{data-substitude}, in diesem Attribut wird der Index, den die Supplierung dieser Stunde im Supplierungenarray hat, gespeichert. Diese Information wird beim öffnen der Popup-Fenster benötigt.\\

\paragraph{News\\}
Bei der Newsseite handelt es sich um eine leere HTML-Seite mit der Überschrift News. Die Neuigkeiten werden gleich wie die Stundenplan- und Supplierplandaten vom Server heruntergeladen und einfach auf die Seite geschrieben. Dabei muss lediglich auf die Formatierung geachtet werden, da \enquote{$\backslash$n} durch ein \enquote{<br>} ersetzt werden muss.\\

\paragraph{Popup-Fenster\\}
Bei den Popup-Fenstern handelt es sich um ein kleines Fenster, welches erscheint wenn auf eine Zeile der Stundenplan- oder Supplierplantabelle gedrückt wird. Darin werden weitere Informationen zu der ausgewählten Unterrichtsstunde oder Supplierung angegeben.\\
Um die richtigen Informationen in das Popup zu schreiben, müssen diese zuerst aus dem Stundenplandatenobjekt, welches beim laden der Daten aus der Datenbank erzeugt wurde, herausgefiltert werden. Um nur die relevanten Daten zu bekommen, wird das Objekt in einer Schleife durchgearbeitet und wenn die ID der Zelle mit der Variable \enquote{cellid} aus dem Objekt übereinstimmt werden die Daten in das Fenster geschrieben.\\
Der gleiche Vorgang wird auf das Popup beim Supplierplan angewandt.\\
Bei den Popups beim angepassten Stundenplan muss zusätzlich noch überprüft werden, ob es sich um eine supplierte Stunde handelt, dazu wird einfach überprüft ob das Attribut \enquote{data-substitude} in der ausgewählten Zelle gesetzt ist. Handelt es sich nicht um eine supplierte Stunde wird das Popup gleich generiert wie beim normalen Stundenplan. Ist aber das Attribut \enquote{data-substitude} gesetzt wird zuerst das normale Popup generiert, danach werden noch die Supplierungsdaten aus dem Supplierplandatenobjekt (Indes ist im Attribut gespeichert) gelesen und der Inhalt des Popup-Fensters wird dementsprechend verändert.\\



Das Monitorformular wurde wegen seines besondern Aufbaus nicht über die bereits bestehende Bibliothek für Formulare erstellt.\\
Prinzipiell wurde in zwei Teile unterschieden.\\
Einerseits gibt es im oberen Teil der Seite eine Liste aller aktiver, registrierter Monitore mit ihren Eigenschaften (Zeit der Registrierung, Name, Raum, Abteilung, Display-Modus, Status des Monitors, IP-Adresse). Dazu gibt es bei jedem Monitor einen Haken, mit dem er zum Bearbeiten ausgewählt werden kann. Zusätzlich gibt es einen Button \enquote{Auswahl invertieren}, über den alle Buttons, die nicht gesetzt sind, gesetzt werden, und umgekehrt. Dies soll die Auswahl erleichtern.\\
Im unteren Teil der Seite können die Eigenschaften, der ausgewählten Monitore verändert werden. Wird eine Eigenschaft leer gelassen, so wird für diese der alte Wert des jeweiligen Monitors beibehalten.
\subsection{Token-Generator}
\label{sec:content_draft_token}
Um der unter \gref{sec:content_security_xsrf} beschriebenen Gefahr entgegenzuwirken, wurde eine PHP-Klasse erstellt, die Methoden zum Erstellen und Auswerten von Formular-Tokens bereitstellt.\\
Grundsätzlich ist hierbei folgendermaßen vorzugehen:
\begin{itemize}
	\item Bereits beim Initialisieren des Objektes wird der Constructor-Methode der Name des Formulars und eine eindeutige ID des Formulars (zum Beispiel der Dateiname) mitgegeben.
	\item Die Generierungs-Methode generiert die internen Schlüssel, die zum Identifizieren des Formulares nötig sind.
	\item Die Ausgabe-Methode gibt den HTML-Code für die versteckten Formular-Elemente aus, die nötig sind, um die Session zu identifizieren.
	\item Die Prüfmethode stellt sicher, ob das abgesendete Formular zuvor für den Benutzer generiert wurde und wirft gegebenenfalls eine Exception.
\end{itemize}
Vereinfacht gesagt, wird ein zufallsgenerierter Hash versteckt in das Formular eingebunden, so dass er beim Absenden des Formulars mitgesendet wird. Dieser Hash wird auch in der PHP-Session gespeichert und nach dem Absenden des Formulars am Server mit dem gespeicherten Wert verglichen.\\
Für die genaue Umsetzung dieser Klasse und ihre Einbindung in Formulare, siehe \gref{sec:content_imple_hashgenerator}
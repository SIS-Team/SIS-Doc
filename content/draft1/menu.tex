\subsection{Menü-Generierung}
Die Generierung des Hauptmenüs erfolgt nicht über das Design-Management, sondern erfolgt in einem mehr oder weniger autonomen System. Dieses System ist als eine Art Library zu verstehen, der man seine Parameter für jeden der 8 möglichen Menüpunkte übergibt (siehe \gref{sec:content_draft_design_web}).\\
Mögliche Parameter sind:
\begin{itemize}
	\item Soll der Menüpunkt angezeigt werden?
	\item Soll der Menüpunkt wählbar sein? (Dies wird benötigt, da manche Benutzer zwar ein Menü einsehen, jedoch nicht alle Punkte des Menüs verwenden dürfen.)
	\item die Grafik, die auf dem Button abgebildet sein soll
	\item der Text unterhalb des Buttons
	\item die URL auf die bei Klick verwiesen werden soll, falls JavaScript aktiviert ist
	\item die URL auf die bei Klick verwiesen werden soll, falls JavaScript deaktiviert ist
\end{itemize}
Zusätzlich wird noch angegeben:
\begin{itemize}
	\item der Name des Benutzers
	\item der Name des aktuellen Menüs
	\item die URL für den Zurück-Button
\end{itemize}
Aus diesen Parametern wird vom Menü-System das fertige Menü generiert.\\
Die notwendigen Dateien sind unter /modules/menu/ abgelegt.
\subsection{Dateibaum}
\label{sec:files}
Der Aufbau der Ordnerstruktur für das Projekt wird folgendermaßen festgelegt.

\begin{description}[style=nextline]
	\item[/backend/]
		Hier befindet sich das Menü für die Eingaben.\\
		Dieses Menü ist für die Benutzergruppen \enquote{SIS-Admin-E}, \enquote{SIS-Admin-N}, \enquote{SIS-Admin-M}, \enquote{SIS-Admin-W}, \enquote{SIS-News} und \enquote{SIS-SuperUser} verfügbar.
		\begin{description}[style=nextline]
			\item[./absentees/]
				Hier befindet sich das Menü für das Eintragen der Fehlenden.\\
				Dieses Menü ist für die Benutzergruppen \enquote{SIS-Admin-E}, \enquote{SIS-Admin-N}, \enquote{SIS-Admin-M}, \enquote{SIS-Admin-W} und \enquote{SIS-SuperUser} verfügbar.
				\begin{description}[style=nextline]
					\item[./classes/]
						Hier können fehlende Klassen eingetragen werden.\\
						Dieses Formular ist für die für die Benutzergruppen \enquote{SIS-Admin-E}, \enquote{SIS-Admin-N}, \enquote{SIS-Admin-M}, \enquote{SIS-Admin-W} und \enquote{SIS-SuperUser} verfügbar.
					\item[./teachers/]
						Hier können fehlende Lehrer eingetragen werden.\\
						Dieses Formular ist für die für die Benutzergruppen \enquote{SIS-Admin-E}, \enquote{SIS-Admin-N}, \enquote{SIS-Admin-M}, \enquote{SIS-Admin-W} und \enquote{SIS-SuperUser} verfügbar.
				\end{description}
				\item[./administration/]
					Hier befindet sich das Administratoren-Menü.\\
					Dieses Menü sowie alle Unterpunkte sind nur für die Benutzergruppe \enquote{SIS-SuperUser} verfügbar.
					\begin{description}[style=nextline]
						\item[./classes/]
							In diesem Formular können die Klassen modifiziert werden.
						\item[./hours/]
							Hier können die zeitlichen Unterrichtsstunden verändert werden.
						\item[./lessons/]
							In diesem Formular können die Stundenpläne verändert werden.
						\item[./rooms/]
							Hier können die Räume eingetragen werden.
						\item[./sections/]
							Hier können die Abteilungen modifiziert werden.
						\item[./subjects/]
							Fächer können hier hinzugefügt werden.
						\item[./teachers/]
							In diesem Formular können die Lehrer modifiziert werden.
					\end{description}
				\item[./monitors/]
					Alle Einstellungen für die Monitore ind hier zu finden.\\
					Dieses Formular ist für die für die Benutzergruppen \enquote{SIS-Admin-E}, \enquote{SIS-Admin-N}, \enquote{SIS-Admin-M}, \enquote{SIS-Admin-W} und \enquote{SIS-SuperUser} verfügbar.
				\item[./news/]
					In diesem Punkt können News eingetragen werden.\\
					Dieses Menü ist für die Benutzergruppen \enquote{SIS-Admin-E}, \enquote{SIS-Admin-N}, \enquote{SIS-Admin-M}, \enquote{SIS-Admin-W}, \enquote{SIS-News} und \enquote{SIS-SuperUser} verfügbar.
				\item[./substitudes/]
					Hier befindet sich das Menü für die Auswahl der Abteilung bei den Supplierungen.\\
					Dieses Menü ist nur für die Benutzergruppe \enquote{SIS-SuperUser} verfügbar.
					\begin{description}[style=nextline]
						\item[./form/]
							Hier können die Supplierungen eingetragen werden.\\
							Dieses Formular ist nur für die Benutzergrupppen \enquote{SIS-Admin-E}, \enquote{SIS-Admin-N}, \enquote{SIS-Admin-M}, \enquote{SIS-Admin-W} und \enquote{SIS-SuperUser} verfügbar.
					\end{description}
		\end{description}	
	\item[/cookies/]
			Hier müssen vor dem Betreten der Seite die Cookies akzeptiert werden (siehe %\ref{sec_cookies}).\\
		\item[/data/]
			\begin{description}[style=nextline]
				\item[./fonts/]
					Alle Schriftart-Dateien liegen hier.
				\item[./images/]
					Hier befinden sich sämtliche Bilder.\\
					Dateien, welche sich logisch gruppieren lassen, besitzen eigene Ordner.
				\item[./scripts/]
					Alle Javascripts, welche in Dateien extrahiert wurden, sind hier zu finden.
				\item[./styles/]
					Hier sind sämtliche Stylesheets.	
			\end{description}
		\item[/impressum/]
			Wie der Name schon sagt, ist hier das Impressum zu finden.
		\item[/login/]
			Hier befindet sich der Login.
		\item[/logout/]
			Und hier ist der Logout.
		\item[/logs/]
			Hier sind spezielle Log-Dateien zu finden.
		\item[/mobile/]
			Die Mobil-Seite befindet sich hier.
			\begin{description}[style=nextline]
				\item[./api/]
					Hier ist die API für die App hinterlegt.
			\end{description}
		\item[/modules/]
			Hier sind die verschiedenen Module des Systems bzw. ihre Komponenten  gespeichert.
			\begin{description}[style=nextline]
				\item[./datebase/]
					Dateien für den allgemeinen Zugriff auf die Datenbank werden hier gespeichert.
				\item[./design/]
					Die HTML-Dateien der Designs werden hier abgelegt.
				\item[./external/]
					Fremdmodule werden hierher kopiert.
				\item[./form/]
					Hier finden sich Dateien mit Funktionen für den Aufbau von HTML-Formularen.
				\item[./general/]
					Allgemeine wichtige Module sind hier untergebracht. U.a. findet man hier die Dateien für den Zugriff auf die Datenbank, das Session-Management oder den Zugriff auf LDAP.
				\item[./menu/]
					Die Dateien in diesem Ordner dienen zum Generieren der Haupt-Menüstruktur.
				\item[./monitors/]
					Dateien, die das Monitorsystem betreffen sind hier zu finden.
				\item[./other/]
					Alle Module, die sich nicht einordnen lassen, werden hier untergebracht.
			\end{description}
		\item[/monitors/]
			Hier ist der Startpunkt für das Monitorsystem.
			\begin{description}[style=nextline]
				\item[./api/]
					Hier befindet sich die API für die Monitore.
				\item[./media/]
					Inhaltbezogene Mediendateien (Bilder und Videos) für das Monitorsystem werden hier gespeichert.
			\end{description}
		\item[/news/]
			Die Website für die News liegt hier.
		\item[/pdf/]
			Hier werden die Druckversionen als PDF zur Verfügung gestellt.
		\item[/substitudes/]
			Die Supplierpläne werden hier generiert.
		\item[/timetables/]
			Der eigene Stundenplan wird hier generiert.
			\begin{description}
				\item[./all/]
					Lehrer können hier die Klassenstundenpläne ansehen.
			\end{description}
		\item[/tmp/]
			Ordner für temporäre Dateien.
\end{description}

Zusätzlich gibt es im Projekt ein Verzeichnis /raspberryConfig/. Hier werden Konfigurationsdateien für die Raspberry Pis hinterlegt.
\subsubsection{AJAX (Buchberger)} 
\label{sec:content_js_AJAX}
\begin{quote}
\enquote{Ajax steht für Asynchronous Javascript + XML. Der Name ist - geben wir es ehrlich zu - ein wenig künstlich.} \cite[S. 317]{ajax}
\end{quote}
Ajax stellt eine Möglichkeit dar, zur JavsScript-Laufzeit ohne Neuladen der Seite Daten vom Server anzufragen.\\
Technisch gesehen ist Ajax ein JavaScript Objekt (\texttt{XMLHttpRequest()}), das vom Webbrowser zur Verfügung gestellt wird. Dieses Objekt kann über bestimmte Methoden HTTP-Anfragen senden.\\
Prinzipiell gibt es im \texttt{XMLHttpRequest()}-Objekt Möglichkeiten für synchrone Anfragen (Die JavaScript-Ausführung bleibt stehen, und wird erst fortgesetzt, wenn die Antwort ankommt.) und asynchrone Anfragen (Ausführung läuft weiter. Es wird jedoch eine Funktion angegeben, die ausgeführt wird, wenn die Antwort ankommt.).\\
\begin{lstlisting}[style=custom, language=JavaScript,  caption={Synchrones AJAX},label={lst:content_ajax_0}]
var http = new XMLHttpRequest();
// Der dritte Parameter der Methode open bestimmt, ob die Anfrage asynchron sein soll.

// synchrone Anfrage
http.open("GET", "file.php", false);

// Sendet Request ab.
http.send(null);

// Gibt den HTTP-Status der Antwort aus.
console.log(http.status);
// Gibt den HTTP-Körper der Antwort aus.
console.log(http.responseText);
\end{lstlisting}
\begin{quote}
\enquote{
\texttt{
readyState:\\
Holds the status of the XMLHttpRequest. Changes from 0 to 4:\\
0: request not initialized\\
1: server connection established\\
2: request received\\
3: processing request\\ 
4: request finished and response is ready}}\\
\cite[\href{http://www.w3schools.com/ajax/ajax_xmlhttprequest_onreadystatechange.asp}{http://www.w3schools.com/ajax/ajax\_xmlhttprequest\_\\onreadystatechange.asp}]{w3schools}
\end{quote}
\begin{lstlisting}[style=custom, language=JavaScript,  caption={Asynchrones AJAX},label={lst:content_ajax_1}]
var http = new XMLHttpRequest();
// Der dritte Parameter der Methode open bestimmt, ob die Anfrage asynchron sein soll.

// asynchrone Anfrage
http.open("GET", "file.php", true);

// Sendet Request ab.
http.send(null);

http.onreadystatechange = function() {
	if (http.readyState == 4) {
		// Gibt den HTTP-Status der Antwort aus.
		console.log(http.status);
		// Gibt den HTTP-Körper der Antwort aus.
		console.log(http.responseText);
	}
}
\end{lstlisting}
GET-Parameter können einfach an den Dateinamen angehängt werden.\\
POST-Parameter sind nicht ganz so einfach mitzugeben, da diese im Körper der HTTP-Anfrage stehen, dementsprechend muss die Länge mitgesendet werden.
\begin{lstlisting}[style=custom, language=JavaScript,  caption={Synchrones AJAX mit POST-Parameter},label={lst:content_ajax_2}]
var http = new XMLHttpRequest();
// Der dritte Parameter der Methode open bestimmt, ob die Anfrage asynchron sein soll.

// synchrone Anfrage
http.open("GET", "file.php", false);

var parameter = "hallo=welt";

// Setzen der nötigen Header.
http.setRequestHeader("Content-type", "application/x-www-form-urlencoded");
// Länge des HTTP-Körpers.
http.setRequestHeader("Content-length", parameter.length);
http.setRequestHeader("Connection", "close");

// Sendet Request mit Parameter ab.
http.send(parameter);

// Gibt den HTTP-Status der Antwort aus.
console.log(http.status);
// Gibt den HTTP-Körper der Antwort aus.
console.log(http.responseText);
\end{lstlisting}
\subsection{Denial of Service (Klotz)}
\paragraph{Problem\\}
Als Denial of Service, bezeichnet man die Nichtverfügbarkeit eines Dienstes. Meistens spricht man von DoS wenn Service aufgrund einer Überlastung nicht mehr erreichbar ist. Die Nichtverfügbarkeit kann aber auch andere Gründe haben.\\
In unserem Fall stellt DoS keine allzu große Gefahr dar, da weder in Form von Rechenleistung noch in Form von Datenraten große Kapazitäten benötigt werden.\\

\paragraph{Methoden um Vorzubeugen\\}
Als Sicherheit wurde während des Testbetriebes des Systems der Traffic überprüft und überschlagsmäßig berechnet mit welchen Datenraten zu rechnen ist. Dabei haben wir festgestellt, dass wir mit unseren Datenraten weit unter der Grenze des an unserer Schule möglichen Traffics sind.\\
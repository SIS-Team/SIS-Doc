\subsection{Brute-Force Attack (Weiland)}
\paragraph{Problem}
Bei einer Brute-Force Attacke werden mangels fehlendes Algorithmus solange mögliche Kombinationen ausprobiert, bis ein Erfolg, z.B. Zutritt zu mit Passwort gesichertem Bereich, eintritt.
\subparagraph{dictionary attack}
Bei einem Wörterbuchangriff werden zuerst häufig verwendete Passwörter wie zum Beispiel \enquote{password1!} oder \enquote{admin} ausprobiert. 
\paragraph{Methoden um Vorzubeugen}
Die am leichtesten anzuwendende Methode zum Verhindern bzw. Verlangsamen einer Brute-Force-Attacke ist ein möglichst langes Passwort mit Zahlen, Groß- und Kleinschreibung und Sonderzeichen. Hierfür gibt es mehrere Lösungsansätze, wie zum Beispiel ganze Sätze als Passwörter zu verwenden, da eine normale Brute-Force-Attacke nahezu unendlich lange benötigen würde. Jedoch ist zu beachten, dass für den Fall, dass der Angreifer das richtige Passwort gleich zu beginn oder nach kurzer Zeit probiert auch ein langes Passwort recht schnell geknackt ist, wobei die Wahrscheinlichkeit hierfür mit zunehmender Länge des Passworts sinkt.\\
\\
Mittels eines der im Internet angebotenen Passwortüberprüfer \\( \href{https://review.datenschutz.ch/passwortcheck/check.php}{https://review.datenschutz.ch/passwortcheck/check.php}) ergab zum Beispiel der Satz \enquote{Dieses\_Passwort\_ist\_sogar\_vor\_9\_Idioten\_sicher\!} eine maximale Dauer von\\ 348'140'598'685'071'194'490'295'439'208'633'571'979'278'835'755'974'363'969'507'922'103'075 Jahre, wenn 2 Milliarden Versuche pro Sekunde durchgeführt werden. Nur zum Vergleich: Wissenschaftler sagen die Erde sei ca. 4,6 Milliarden Jahre alt. Das ist ca. um den Faktor 7.6 * 10$^{58}$ weniger.

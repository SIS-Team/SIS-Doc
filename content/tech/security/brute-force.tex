\subsubsection{Brute-Force Attack (Weiland)}
\paragraph{Problem}
Bei einer Brute-Force Attacke werden mangels fehlender Algorithmen solange mögliche Kombinationen ausprobiert, bis ein Erfolg, z.B. Zutritt zu mit Passwort gesichertem Bereich, eintritt.
\subparagraph{dictionary attack}
Bei einem Wörterbuchangriff werden zuerst häufig verwendete Passwörter wie zum Beispiel \enquote{password1!} oder \enquote{admin} ausprobiert. 
\subparagraph{reverse Brute-Force-Attack}
Hierbei wird ein Passwort mit verschiedenen Benutzernamen oder ein Schlüssel auf verschiedene verschlüsselte Daten angewendet, bis ein Vorgang erfolgreich ist.
\paragraph{Methoden um Vorzubeugen}
Eine zum Teil sehr effektive Verhinderung von Brute-Force-Attacken kann dadurch erreicht werden, dass nur eine begrenzte Anzahl von Anmeldungsversuchen  zugelassen ist. Jedoch kann diese Sicherheitsvorkehrung auch umgangen werden. Wird die Anzahl der Anmeldungsversuche pro IP-Adresse gezählt, muss der Angreifer sich nur eine andere IP-Adresse geben, um weitere Versuche zu erhalten. Zudem ist diese Methode nutzlos, wenn der Angreifer die Attacke offline durchführt, wie zum Beispiel beim Versuch einen verschlüsselten Datenverkehr zwischen Client und Server zu entschlüsseln, da die Daten am Rechner des Angreifers liegen und weder der Server, noch der Client dies erkennen können. \\\\
Die am leichtesten anzuwendende Methode zum Verhindern bzw. Verlangsamen einer Brute-Force-Attacke ist ein möglichst langes Passwort mit Zahlen, Groß- und Kleinschreibung und Sonderzeichen. Hierfür gibt es mehrere Lösungsansätze, wie zum Beispiel ganze Sätze als Passwörter zu verwenden, da eine normale Brute-Force-Attacke nahezu unendlich lange benötigen würde. Jedoch ist zu beachten, dass für den Fall, dass der Angreifer das richtige Passwort gleich zu Beginn oder nach kurzer Zeit probiert, auch ein langes Passwort recht schnell geknackt ist, wobei die Wahrscheinlichkeit hierfür mit zunehmender Länge des Passworts sinkt.\\
\\
Mittels eines der im Internet angebotenen Passwortüberprüfer \\( \href{https://review.datenschutz.ch/passwortcheck/check.php}{https://review.datenschutz.ch/passwortcheck/check.php}) ergab zum Beispiel der Satz \enquote{Dieses\_Passwort\_ist\_sogar\_vor\_9\_Idioten\_sicher!} eine maximale Dauer von\\ 348'140'598'685'071'194'490'295'439'208'633'571'979'278'835'755'974'363'969'507'922'103'075 Jahre ($\approx 3,5 \cdot 10^{66}$ Jahre), wenn 2 Milliarden Versuche pro Sekunde durchgeführt werden. Nur zum Vergleich: Wissenschaftler sagen die Erde sei ca. 4,6 Milliarden Jahre alt. Das ist ca. um den Faktor 7.6 $\cdot 10^{58}$ weniger.

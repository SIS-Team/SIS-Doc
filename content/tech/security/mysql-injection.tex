\subsubsection{MySQL-Injection (Handle)}
Mit SQL Injections versucht ein Angreifer Daten aus der Datenbank auszulesen, Daten zu manipulieren und noch einige weitere Dinge, die nicht im Auge des Programmierers liegen, zu machen. Diese Injections sollen natürlich so gut es geht vom Programmierer verhindert werden, dazu stellt PHP einen Befehl zur Verfügung. Dieser lautet: \textit{mysql\_real\_escape\_string()}.
\paragraph{Beispiele}
Bei dem Programm-Code \ref{lst:content_mysql_sql_injection_1} ist gezeigt wie eine Injection erfolgreich durchgeführt werden kann. Dabei wird bei einer einfachen Anmeldung die Anmeldung ausgetrickst und man kann sich ohne Passwort anmelden.
\begin{lstlisting}[style=custom, language=PHP, caption={MySQL Injection: Falsch},label={lst:content_mysql_sql_injection_1}]
<?php 
	$sql = "SELECT * FROM users	WHERE name='$_POST['username']' AND password='$_POST['password']'";	//Die mitgegebenen Werte werden in den Query übernommen
	mysql_query($sql);	//Query wird ausgeführt
	
?>
\end{lstlisting}
Sofern der User wirklich einen regulären Benutzername und einen reguläres Passwort eingibt macht diese Vorgangsweise keine Probleme.\\
Nehmen wir jetzt an, dass der User ein Angreifer ist und sich ohne ein Passwort zu kennen zugriff zum passwortgeschützten Bereich machen will, dann macht der Programm-Code \ref{lst:content_mysql_sql_injection_1} Probleme.\\
Beispiel:\\
\begin{itemize}
	\item \$\_POST['username'] = 'asdf'\\
	Dieser Benutzername muss nicht existieren in der Datenbank
	\item \$\_POST['password'] = "' OR ''=''"\\
	Damit wird ein Ausdruck angefügt der immer 1 ergibt
\end{itemize}
$ \rightarrow $ \textit{SELECT * FROM users WHERE name='asdf' AND password='' OR ''=''}\\
Mit dieser Eingabe eines Angreifers bekommt man bei diesem Select Query immer ein Ergebnis zurück. Welches, kommt drauf an wie es ausgewertet wird, zum erfolgreichen einloggen führen kann. Dies kann mit dem oben erwähnten Befehl verhindert werden. Siehe Programm-Code \ref{lst:content_mysql_sql_injection_2}
\begin{lstlisting}[style=custom, language=PHP, caption={MySQL Injection: Richtig},label={lst:content_mysql_sql_injection_2}]
<?php 
	$user = mysql_real_escape_string($_POST['username']);
	$paswd = mysql_real_escape_string($_POST['password']);
	$sql = "SELECT * FROM users WHERE name='$user' AND password='$paswd'";	//Die kontrollierten Werte werden in den Query Übernommen
	mysql_query($sql);	//Query wird ausgeführt
	
?>
\end{lstlisting}
Geht man davon aus, dass die selben Werte vom Angreifer eingegeben werden, so kommt es jetzt nicht zu einem Ergebnis, da die Funktion alle nicht erlaubten Zeichen 'eliminiert'. Sie werden nicht direkt eliminiert, sondern die nicht erlaubten Zeichen werden nicht gelöscht, sondern mit einem \textbackslash maskiert. Also aus einem \textit{'} wird ein \textit{\textbackslash'}, aus einem \textit{"} wird ein \textit{\textbackslash"} usw. Die Zeichen, die mit einem Backslash maskiert werden, sind ",',\textbackslash n,\textbackslash r,\textbackslash,\textbackslash x00,\textbackslash x1a.\\
Es können nicht nur wie oben gezeigt SELECT-Befehle verändert werden, sondern es können ganze Befehle in Befehle eingefügt werden. Dadurch kann ein Angreifer zum Beispiel einen neuen Datensatz anlegen, Datensätze löschen und so weiter.
\begin{lstlisting}[style=custom, language=PHP, caption={MySQL Injection: Beispiel Query in Qeury},label={lst:content_mysql_sql_injection_3}]
<?php 
	$ID = $_GET['ID'];
	
	$sql = "SELECT * FROM users	WHERE ID={$ID}";
	mysql_query($sql);	//Query wird ausgeführt
	
	//Mitgegebener GET Wert = 2
	//Query : SELECT * FROM users	WHERE ID=2
	
	//Mitgegebener GET WERT = 2;UPDATE users SET type="admin" WHERE ID=23
	//Query : SELECT * FROM users	WHERE ID=2;UPDATE users SET type="admin" WHERE ID=23
	//Damit kann zum Beispiel einem User eine andere Berechtigung gegeben werden
?>
\end{lstlisting}
Im \autoref{lst:content_mysql_sql_injection_3} ist gezeigt wie ein Angreifer zum Beispiel einem beliebigen User die Berechtigungen admin gebem kann. Was natürlich eine sehr große Sicherheitslücke darstellt. Dies kann wiederum mit der Funktion \textit{mysql\_real\_escape\_string()} verhindert werden.